\documentclass[11pt,twoside,onecolumn]{book}
\usepackage[]{fontenc}
\usepackage{palatino}
\usepackage{a4}
\addtolength{\oddsidemargin}{-0.2in}
\addtolength{\evensidemargin}{-0.6in}
\addtolength{\textwidth}{0.5in}

\pagestyle{headings}
\setcounter{secnumdepth}{3}
\setcounter{tocdepth}{3}
\begin{document}

\pagenumbering{roman}
\pagestyle{empty}


\begin{center}

\vfill

{ \Huge
The omniORB2 User's Guide
}

\vfill

{ \Large
Sai-Lai Lo\\
{\normalsize ({\it email: sll@orl.co.uk})}\\
Olivetti \& Oracle Research Laboratory\\
}

\vfill


{\it Note: this document is very incomplete at the moment!
More chapters will be added to document the ORB's APIs and its internals.}

\vfill
\vfill
13 Mar, 1997
\vfill

\end{center}
\cleardoublepage
\tableofcontents
\cleardoublepage
\pagestyle{headings}

\pagenumbering{arabic}

%\input{intro.tex}
\chapter{Introduction}

OmniORB2 is an Object Request Broker (ORB) that implements the 2.0
specification of the Common Object Request Broker Architecture (CORBA)~\cite{corba2-spec}.
This user guide tells you how to use omniORB2 to develop CORBA applications.
It assumes a basic understanding of CORBA. 

In this chapter, we give an overview of the main features of omniORB2 and
what you need to do to setup your environment to run omniORB2.

\section{Features}

\subsection{CORBA 2 compliant}

OmniORB2 implements the Internet Inter-ORB Protocol (IIOP).
This protocol provides omniORB2 the means of achieving interoperability
with the ORBs implemented by other vendors. In fact, this is the
native protocol used by omniORB2 for the communication amongst its objects
residing in different address spaces. Moreover, the IDL to C++ language
mapping provided by omniORB2 conforms to the latest revision of the CORBA
specification.

\subsection{Multithreading}

OmniORB2 is fully multithreaded. To achieve low IIOP call overhead,
unnecessary call-multiplexing is eliminated. At any time, there is at most
one call in-flight in each communication channel between two address
spaces. To do so without limiting the level of concurrency, new channels
connecting the two address spaces are created on demand and cached when
there are more concurrent calls in progress. Each channel is served by a
dedicated thread. This arrangement provides maximal concurrency and
eliminates any thread switching in either of the address spaces to process
a call. Furthermore, to maximize the throughput in processing large call
arguments, large data elements are sent as soon as they are processed while
the other arguments are being marshalled.

\subsection{Portability}

At ORL, the ability to target a single source tree to multiple platforms is
very important. This is difficult to achieve if the IDL to C++ mapping for
these platforms are different. We avoid this problem by making sure that
only one IDL to C++ mapping is used. We run several flavours of Unices, Windows
NT, Windows 95 and our in-house developed systems for our own
hardware. OmniORB2 have been ported to all these platforms. {\bf The IDL to
C++ mapping for these targets are all the same}.

OmniORB2 uses real C++ exceptions and nested classes. We stay with the
CORBA specification's standard mapping as much as possible and do not use
the alternative mappings for C++ dialects. The only exception is the
mapping of {\bf modules} to {\bf C++ classes} instead of {\bf namespaces}.

OmniORB2 relies on the native thread libraries to provide the
multithreading capability. A small class library (omnithread~\cite{tjr96a})
is used to encapsulated the (possibly different) APIs of the native thread
libraries. In the application code, it is recommended but not mandatory to
use this class library for thread management. It should be easy to port
omnithread to any platform that either supports the POSIX thread standard
or has a thread package that supports similar capabilities.

\subsection{Missing features}

OmniORB2 is not (yet) a complete implementation of the CORBA core. 
The following is a list of the missing features. 

\begin{itemize}

\item The Typcode and the Any type is not supported. Support for these
types will be added shortly.
\item The BOA only support the persistent server activation policy. Other
dynamic activation and deactivation policies are not supported.
\item The Dynamic Invocation Interface is not supported.
\item The Dynamic Skeleton Interface is not supported.
\item OmniORB2 does not has its own Interface Repository.
\end{itemize}

\sloppy{These features may be implemented in the short to medium term. It is best
to check out the latest status on the omniORB2 home page ({\tt http://intranet.cam-orl.co.uk/$\sim$omni/}).}


\section{Setting Up Your Environment}

\sloppy{At ORL, you should use the OMNI Development Environment (ODE)~\cite{tjr96b}
and the OMNI tree version 5.0 or above to compile your programs. If this is
the case, there is no extra setup you have to do other than those described
in the ODE documentation.}

If you are running omniORB2 at other sites, you (or your system
administrator) should install omniOB2 by following the instructions in the
installation notes. 

\begin{itemize}

\item On Unix platforms, the omniORB2 runtime looks for the environment
variable {\tt OMNIORB\_CONFIG}. If this variable is defined, it contains
the pathname of the omniORB2 configuration file. If the variable is not
set, omniORB2 will use the compiled-in pathname to locate the file.

\item On ARM/ATMos, the omniORB2 runtime looks for configuration
information in the file {\tt omniORB.cfg}.

\item On Win32 platforms (Windows NT, Windows '95), omniORB2 first checks the 
environment variable ({\tt OMNIORB2\_CONFIG}) to obtain the pathname of the 
configuration file. If this is not set, it then attempts to obtain 
configuration data in the system registry. It searches for the data under 
the key {\tt HKEY\_LOCAL\_MACHINE$\backslash$SOFTWARE$\backslash$ORL$\backslash$omniORB$\backslash$2.0}

\end{itemize}

The configuration file is used to obtain an object reference for the
COSS Naming Service. The entry in the configuration file should be specified
in the following form:

{\tt NAMESERVICE <stringified IOR for the COSS Naming Service>}

Comments in the configuration file should be prefixed with a \#.

On Win32 platforms, the stringified IOR can be placed in the system registry, 
in the (string) value {\tt NAMESERVICE}, under the key 
{\tt HKEY\_LOCAL\_MACHINE$\backslash$SOFTWARE$\backslash$ORL$\backslash$omniORB$\backslash$2.0}.



%\input{basic.tex}
\chapter{The Basics}

In this chapter, we go through three examples to illustrate the practical
steps to use omniORB2. By going through the source code of each example,
the essential concepts and APIs are introduced. If you have no previous
experience with using CORBA, you should study this chapter in detail. There
are pointers to other essential documents you should be familiar with.

If you have experience with using other ORBs, you should still go through
this chapter because it provides important information about the features
and APIs that are necessarily omniORB2 specific. For instance, the object
implementation skeleton is covered in section~\ref{stubobjimpl}.

\section{The Echo Object Example}

Our example is an object which has only one method. The method simply echos
the argument string. We have to:

\begin{enumerate}

\item define the object interface in IDL;
\item use the IDL compiler to generate the stub code\footnote{The stub code
is the C++ code that provides the object mapping as defined in the CORBA 2.0
specification.};
\item provide the object implementation;
\item write the client code.

\end{enumerate}

The source code of this example is included in the last section of this
chapter. A makefile written to be used under the OMNI Development
Environment (ODE)~\cite{tjr96b} is alsoincluded.

\section{Specifying the Echo interface in IDL}

We define an object interface, called Echo, as follows:

{\small
\begin{verbatim}

interface Echo {
    string echoString(in string mesg);
};

\end{verbatim}
}

If you are new to IDL, you can learn about its syntax in Chapter 3 of the
CORBA specification 2.0~\cite{corba2-spec}.

For the moment, you only need to know that the interface consists of a
single operation, echoString, which takes a string as an argument and
returns a copy of the same string.

The interface is written in a file, called {\tt echo.idl}. If you are
using ODE, all IDL files should have the same
extension- {\tt.idl} and should be placed in the {\tt idl} directory of
your export tree. This is done so that the stub code will be generated
automatically and kept up-to-date with your IDL file.

For simplicity, the interface is defined in the global IDL namespace. This
practice should be avoided for the sake of object reusuability. If every
CORBA developer defines their interfaces in the global IDL namespace, there
is a danger of name clashes between two independently defined
interfaces. Therefore, it is better to qualify your interfaces by defining
them inside {\tt module} names. Of course, this does not eliminate the
chance of a name clash unless some form of naming convention is agreed
globally. Nevertheless, a well-chosen module name can help a lot.

\section{Generating the C++ stubs}

From the IDL file, we use the IDL compiler to produce the C++ mapping of
the interface. The IDL compiler for omniORB2 is called {\tt
omniidl2}. Given the IDL file, {\tt omniidl2} produces two stub files: a
C++ header file and a C++ source file. For example, from the file {\tt
echo.idl}, the following files are produced:

\begin{itemize}
\item {\tt echo.hh}
\item {\tt echoSK.cc}
\end{itemize}

If you are using ODE, you don't need to invoke omniidl2 explicitly. In
the example file {\tt dir.mk}, we have the following line:

{\small
\begin{verbatim}

CORBA_INTERFACES = echo

\end{verbatim}
}

That is all we need to instruct ODE to generate the stubs. Remember, you
won't find the stubs in your working directory because all stubs are written
into the {\tt stub} directory at the top level of your build tree. 

\section{A Quick Tour of the C++ stubs}

The C++ stubs conform to the mapping defined in the CORBA 2.0 specification
(chapter 16-18). It is important to understand the mapping before you start
writing any serious CORBA applications.

Before going any further, it is worth knowing what the mapping looks like.

\subsection{Object Reference}

The use of an object interface denotes an object reference. For the
example interface Echo, the C++ mapping for its object reference is {\tt
Echo\_ptr}. The type is defined in echo.hh. The relevant section of the
code is reproduced below:

{\small
\begin{verbatim}

class Echo;
typedef Echo* Echo_ptr;

class Echo : public virtual omniObject, public virtual CORBA::Object {
public:

  virtual char *  echoString ( const char *  mesg ) = 0;
  static Echo_ptr _nil();
  static Echo_ptr _duplicate(Echo_ptr);
  static Echo_ptr _narrow(CORBA::Object_ptr);

  ... // methods generated for internal use
};

\end{verbatim}
}

In a compliant application, the operations defined in an object interface
should {\bf only} be invoked via an object reference. This is done by using
arrow (``$\rightarrow$'') on an object reference. For example, the call to the
operation {\tt echoString} would be written as {\tt obj$\rightarrow$echoString(mesg)}. 

It should be noted that the concrete type of an object reference is opaque,
i.e. you must not make any assumption about how an object reference is
implemented. In our example, even though {\tt Echo\_ptr} is implemented as
a pointer to the class {\tt Echo}, it should not be used as a C++ pointer,
i.e. conversion to void*, arithmetic operations, and relational operations,
including test for equality using {\bf operation==} must not be performed
on the type.

In addition to {\tt echoString}, the mapping also defines three static
member functions in the class Echo: {\tt \_nil}, {\tt \_duplicate}, and 
{\tt \_narrow}. Note that these are operations on an object reference. 

The {\tt \_nil} function returns a nil object reference of the Echo interface. The
following call is guaranteed to return TRUE:

{\small
\begin{verbatim}

CORBA::Boolean true_result = CORBA::is_nil(Echo::_nil());

\end{verbatim}
}

Remember, {\tt CORBA::is\_nil()} is the only compliant way to check if an
object reference is nil. You should not use the equality operator==.

The {\tt \_duplicate} function returns a new object reference of the Echo
interface. The new object reference can be used interchangeably with the old
object reference to perform an operation on the same object.


\sloppy{All CORBA objects inherit from the generic object {\tt CORBA::Object}.
{\tt CORBA::Object\_ptr} is the object reference for {\tt CORBA::Object}.
Any object reference is therefore conceptually inherited from 
{\tt CORBA::Object\_ptr}. In other words, an object reference
such as {\tt Echo\_ptr} can be used in places where a {\tt
CORBA::Object\_ptr} is expected.}

The {\tt \_narrow} function takes an argument of the type {\tt
CORBA::Object\_ptr} and returns a new object reference of the Echo
interface.  If the actual (runtime) type of the argument object reference
can be widened to {\tt Echo\_ptr}, {\tt \_narrow} will return a valid object
reference. Otherwise it will return a nil object reference.

To indicate that an object reference will no longer be accessed, you can
call the {\tt CORBA::release} operation. Its signature is as follows:

{\small
\begin{verbatim}

class CORBA {
   static void release(CORBA::Object_ptr obj);
   ... // other methods
};
\end{verbatim}
}

You should not use an object reference once you have called {\tt
CORBA::release}. This is because the associated resources may have been
deallocated. Notice that we are referring to the resources associated with
the object reference and {\bf not the object implementation}. Here is a
concrete example, if the implementation of an object resides in a different
address space, then a call to {\tt CORBA::release} will only caused the
resources associated with the object reference in the current address space
to be deallocated. The object implementation in the other address space is
unaffected.

As described above, the equality operator== should not be used on object
references. To test if two object references are equivalent, the member
function {\tt \_is\_equivalent} of the generic object {\tt CORBA::Object}
can be used. Here is an example of its usage:

{\small
\begin{verbatim}

Echo_ptr A;
...            // initialized A to a valid object reference 
Echo_ptr B = A;
CORBA::Boolean true_result = A->_is_equivalent(B); 
// Note: the above call is guaranteed to be TRUE

\end{verbatim}
}

You have now been introduced to most of the operations that can be invoked
via {\tt Echo\_ptr}. The generic object {\tt CORBA::Object} provides a few more
operations and all of them can be invoked via {\tt Echo\_ptr}. These operations
deal mainly with CORBA's dynamic interfaces. You do not have to understand
them in order to use the C++ mapping provided via the stubs. For details,
please read the CORBA specification~\cite{corba2-spec} chapter 17.

Since object references must be released explicitly, their usage is prone
to error and can lead to memory leakage. The mapping defines the {\bf
object reference variable} type to make life easier. In our example, the
variable type {\tt Echo\_var} is defined\footnote{In omniORB2, all object
reference variable types are instantiated from the template type
\_CORBA\_ObjRef\_Var.}.

The {\tt Echo\_var} is more convenient to use because it will automatically
release its object reference when it is deallocated or when assigned a new
object reference. For many operations, mixing data of type {\tt Echo\_var} and
{\tt Echo\_ptr} is possible without any explicit operations or castings
\footnote{However, the implementation of the type conversion operator()
between Echo\_var and Echo\_ptr varies slightly among different C++
compilers, you may need to do an explicit casting when the compiler
complains about the conversion being ambiguous.}. For instance, the
operation {\tt echoString} can be called using the arrow
(``$\rightarrow$'') on a {\tt Echo\_var}, as one can do with a {\tt Echo\_ptr}.

The usage of {\tt Echo\_var} is illustrated below:

{\small
\begin{verbatim}

Echo_var a;
Echo_ptr p = ...     // somehow obtain an object reference

a = p;               // a assumes ownership of p, must not use p anymore

Echo_var b = a;      // implicit _duplicate

p = ...              // somehow obtain another object reference

a = Echo::_duplicate(p);     // release old object reference
                             // a now holds a copy of p.
\end{verbatim}
}

\subsection{Object Implementation}
\label{stubobjimpl}

Unlike the client side of an object, i.e. the use of object references, the
CORBA specification 2.0 deliberately leave many of the necessary
functionalities to implement an object unspecified. As a consequence, it is
very unlikely the implementation code of an object on top of two different
ORBs can be identical. However, most of the code are expected to be
portable. In particular, the body of an operation implementation can
normally be ported with no or little modification.

OmniORB2 uses C++ inheritance to provide the skeleton code for 
object implementation. For each object interface, a skeleton class is
generated. In our example, the skeleton class {\tt \_sk\_Echo} is generated for
the Echo IDL interface. An object implementation can be written by creating
an implementation class that derives from the skeleton class. 

The skeleton class {\tt \_sk\_Echo} is defined in {\tt echo.hh}. The
relevant section of the code is reproduced below. 

{\small
\begin{verbatim}

class _sk_Echo :  public virtual Echo {
public:
  _sk_Echo(const omniORB::objectKey& k);
  virtual char *  echoString ( const char *  mesg ) = 0;
  Echo_ptr        _this();
  void            _obj_is_ready(BOA_ptr);
  void            _dispose();
  BOA_ptr         _boa();
  omniORB::objectKey _key();  
  ... // methods generated for internal use
};

\end{verbatim}
}

The code fragment shows the only member functions that can be used in the
object implementation code. Other member functions are generated for
internal use only. {\bf Unless specified otherwise, the description below
is omniORB2 specific.}  The functions are:

\begin{description}

\item[echoString] it is through this abstract function that an
implementation class provides the implementation of the {\tt echoString}
operation. Notice that its signature is the same as the {\tt echoString}
function that can be invoked via the {\tt Echo\_ptr} object reference.
{\bf The signature of this function is specified by the CORBA specfication}.

\item[\_this] this function returns an object reference for the target
object. The returned value must be deallocated via {\tt CORBA::release}.
See~\ref{objeg1} for an example of how this function is used.

\item[\_obj\_is\_ready] this function tells the Basic Object
Adaptor\footnote{The interface of a BOA is described in chapter 8 of the
CORBA specification.} (BOA) that the object is ready to serve. Until this
function is called, the BOA would not serve any incoming calls to this
object. See~\ref{objeg1} for an example of how this function is used.

\item[\_dispose] this function tells the BOA to dispose of the object.
The BOA will stop serving incoming calls of this object and remove any
resources associated with it.
See~\ref{objeg1} for an example of how this function is used.

\item[\_boa] this function returns a reference to the BOA that serves this
object.

\item[\_key] this function returns the key that the ORB used to identify
this object. The type {\tt omniORB::objectKey} is opaque to application
code. The function {\tt omniORB::keyToOctetSequence} can be used to
convert the key to a sequence of octets.

\end{description}

\section{Writing the object implementation}
\label{objimpl}

You define an implementation class to provide the object
implementation. There is little constraint on how you design your
implementation class except that it has to inherit from the stubs' skeleton
class and to implement all the abstract functions defined in the skeleton
class. Each of these abstract functions corresponds to an operation of the
interface. They are hooks for the ORB to perform upcalls to your
implementation.

Here is a simple implementation of the Echo object.

{\small
\begin{verbatim}
class Echo_i : public virtual _sk_Echo {
public:
  Echo_i() {}
  virtual ~Echo_i() {}
  virtual char * echoString(const char *mesg);
};

char *
Echo_i::echoString(const char *mesg) {
  char *p = CORBA::string_dup(mesg);
  return p;
}
\end{verbatim}
}

There are three points to note here:
\begin{description}
\item[Storage Responsibilities] A string, which is used as an IN argument
and the return value of {\tt echoString}, is a variable size data
type. Other examples of variable size data types include sequences, type
``any'', etc. For these data types, you must be clear about who's
responsibility to allocate and release their associated storage. As a rule
of thumb, the client (or the caller to the implementation functions) owns
the storage of all IN arguments, the object implementation (or the callee)
must copy the data if it wants to retain a copy. For OUT arguments and
return values, the object implementation allocates the storage and passes
the ownership to the client. The client must release the storage when the
variables will no longer be used.  For details, please refer to Table 24-27
of the CORBA specification.

\item[Multi-threading] As omniORB2 is fully multithreaded, multiple threads
may perform the same upcall to your implementation concurrently. It is up
to your implementation to synchronise the threads' accesses to shared data.
In our simple example, we have no shared data to protect so no
thread synchronisation is necessary.

\item[Instantiation] You must not instantiate an implementation as
automatic variables. Instead, you should always instantiate an
implementation using the new operator, i.e. its storage is allocated on the
heap. The reason behind this restriction will become clear in
section~\ref{objeg1}.

\end{description}

\section{Writing the client}

Here is an example of how a {\tt Echo\_ptr} object reference is used.

{\small
\begin{verbatim}
void
hello(CORBA::Object_ptr obj)
{
  Echo_var e = Echo::_narrow(obj);                // line 1

  if (CORBA::is_nil(e)) {                         // line 2
    cerr << "hello: cannot invoke on a nil object reference.\n" << endl;
    return;
  }

  CORBA::String_var src = (const char*) "Hello!";  // line 3
  CORBA::String_var dest;                          // line 4

  dest = e->echoString(src);                       // line 5

  cerr << "I said,\"" << src << "\"."
       << " The Object said,\"" << dest <<"\"" << endl;
}
\end{verbatim}
}

Briefly, the function {\tt hello} accepts a generic object reference. The
object reference ({\tt obj}) is narrowed to {\tt Echo\_ptr}. If the object
reference returned by {\tt Echo::\_narrow} is not nil, the operation {\tt
echoString} is invoked. Finally, both the argument to and the return value of
{\tt echoString} are printed to cerr.

The example also illustrates how T\_var types are used. As it was explained
in the previous section, T\_var types take care of storage allocation and
release automatically when variables of the type are assigned to or when
the variables go out of scope. 

In line 1, the variable {\tt e} takes over the storage responsibility of
the object reference returned by {\tt Echo::\_narrow}. The object reference
is released by the destructor of {\tt e}. It is called automatically when
the function returns. Line 2 and 5 shows how a {\tt Echo\_var} variable is
used. As said earlier, {\tt Echo\_var} type can be used interchangeably with
{\tt Echo\_ptr} type.

The argument and the return value of {\tt echoString} are stored in {\tt
CORBA::String\_var} variable {\tt src} and {\tt dest} respectively. The
strings managed by the variables are deallocated by the destructor of
{\tt CORBA::String\_var}. It is called automatically when the function
returns. Line 5 shows how {\tt CORBA::String\_var} variables are used. They
can be used in place of a string (for which the mapping is {\tt char*
})\footnote{A conversion operator() of CORBA::String\_var converts a
CORBA::String\_var to a char*.}. As used in line 3, assigning a constant
string ({\tt const char*}) to a {\tt CORBA::String\_var} causes the string
to be copied. On the otherhand, assigning a {\tt char*} to a {\tt
CORBA::String\_var}, as used in line 5, causes the latter to assume
the ownership of the string\footnote{Please refer to the CORBA
specification 16.7 for the details of the String\_var mapping. Other T\_var
types are also covered in chapter 16.}. 

Under the C++ mapping, T\_var types are provided for all the non-basic data
types.  It is obvious that one should use automatic variables whenever
possible both to avoid memory leak and to maximize performance. However,
when one has to allocate data items on the heap, it is a good practice to
use the T\_var types to manage the heap storage.

\section{Example 1 - Colocated Client and Implementation}
\label{objeg1}

Having introduced the client and the object implementation, we can now
describe how to link up the two via the ORB. In this section, we describe
an example in which both the client and the object implementation are in
the same address space. In the next two sections, we shall describe the
case where the two are in different address spaces.

The code for this example is reproduced below:

{\small
\begin{verbatim}
int
main(int argc, char **argv)
{
  CORBA::ORB_ptr orb = CORBA::ORB_init(argc,argv,"omniORB2");   // line 1
  CORBA::BOA_ptr boa = orb->BOA_init(argc,argv,"omniORB2_BOA"); // line 2

  Echo_i *myobj = new Echo_i();                                 // line 3
  myobj->_obj_is_ready(boa);                                    // line 4

  boa->impl_is_ready(0,1);                                      // line 5

  Echo_ptr myobjRef = myobj->_this();                           // line 6
  hello(myobjRef);                                              // line 7
  CORBA::release(myobjRef);                                     // line 8

  myobj->_dispose();                                            // line 9
  return 0;
}
\end{verbatim}
}

The example illustrates several important interactions among the ORB, the
object implementation and the client. Here are the details:

\subsection{ORB/BOA initialisation}

\begin{description}

\item[line 1] The ORB is initialised by calling the {\tt CORBA::ORB\_init} function. The function uses the 3rd argument to determine which ORB should be
returned. To use omniORB2, this argument must either be ``omniORB2'' or
NULL. If it is NULL, there must be an argument, -ORBid ``omniORB2'', in
{\tt argv}. Like any command-line arguments understood by the ORB, it will
be removed from argv when {\tt CORBA::ORB\_init} returns. Therefore, an
application is not required to handle any command-line arguments it does
not understand. If the ORB identifier is not ``omniORB2'', the
initialisation will fail and a nil {\tt ORB\_ptr} will be returned. If
supplied, omniORB2 also reads the configuration file {\tt
omniORB.cfg}. Among other things, the file provides a list of initial
object references. One example of these object references is the naming
service. Its use will be discussed in section~\ref{resolveinit}. If any
error occurs during the processing of the configuration file, the system
exception CORBA::INITIALIZE is raised.


\item[line 2] The BOA is initialised by calling the ORB's {\tt BOA\_init}.
The 3rd argument must either be ``omniORB2\_BOA'' or NULL. If it
is NULL, then {\tt argv} must contain an argument, -BOAid
``omniORB2\_BOA''. If the BOA identifier is not ``omniORB2\_BOA'', the
initialisation will fail and a nil BOA\_ptr will be returned. Like
{\tt ORB\_init}, any command-line arguments understood by {\tt BOA\_init}
will be removed from {\tt argv}.

\end{description}

\subsection{Object initialisation}

\begin{description}
\item[line 3] An instance of the Echo object is initialised using the {\tt
new} operator.
\item[line 4] The object's {\tt \_obj\_is\_ready} is called. 
This function informs the BOA that this object is ready to serve. Until
this function is called, the BOA will not accept any invocation on the
object and will not perform any upcall to the object.
\item[line 5] The BOA's {\tt impl\_is\_ready} is called. This function tells
the BOA the implementation is ready. After this call, the BOA will accept
IIOP requests from other address spaces. There are 2 points to note here:
\begin{enumerate}
\item {\tt boa$\rightarrow$impl\_is\_ready} can be called any time after
{\tt BOA\_init} is called (line 2). In other words, object instances can be
initialised and advertised to the BOA before or after this function is
called. 
\item The 2nd argument\footnote{The 1st argument is a pointer to the
implementation definition and is always ignored by omniORB2.} to {\tt
impl\_is\_ready} tells the ORB whether this call should be
non-blocking. The default value of this argument is FALSE(0) and the call
will block indefinitely within the ORB. If there are more things the main
thread should do after it calls {\tt impl\_is\_ready}, as it is the case in
this example, the non-blocking option (TRUE=1) should be specified. Whether
the main thread blocks in this call or not, the ORB is not affected because
its functions are provided by other threads spawned internally.  Notice
that the signature of {\tt impl\_is\_ready} in the CORBA specification does
not have the 2nd argument\footnote{The CORBA specification does not specify
when {\tt impl\_is\_ready} should return. Many ORB vendors choose to
implement {\tt impl\_is\_ready} as blocking until a certain time-out value
is exceeded. In a single threaded implementation this is necessary to give
the ORB the time to serve incoming requests.}. Therefore, calling
{\tt impl\_is\_ready} with the non-blocking option is omniORB2 specific.
\end{enumerate}
\end{description}

\subsection{Client invocation}

\begin{description}

\item[line 6] The object reference is obtained from the implementation by
calling {\tt \_this}. Like any object reference, the return value of \_this
must be released by {\tt CORBA::release} when it is no longer needed.
\item[line 7] Call {\tt hello} with this object reference. The argument is
widened implicitly to the generic object reference {\tt CORBA::Object\_ptr}.
\item[line 8] Release the object reference.
\end{description}

One of the important characteristic of an object reference is that it is
completely location transparent. A client can invoke on the object using
its object reference without any need to know whether the object is
colocated in the same address space or resided in a different address
space. 

In case of colocated client and object implementation, omniORB2 is able to
short-circuit the client calls to direct calls on the
implementation methods. The cost of an invocation is reduced to that of a
function call. This optimisation is applicable {\bf not only} to object
references returned by the \_this function but to any object references
that are passed around within the same address space or received from other
address spaces via IIOP calls.

\subsection{Object disposal}

\begin{description}

\item[line 9] To dispose of an object implementation and release all the
resources associated with it, the {\tt \_dispose} function is called.  In
fact, this is the {\bf only} clean way to get rid of an object
implementation. Even though the object is created using the new operator in
the application code, the application should never call the delete operator
on the object directly.

\end{description}

Once an application calls {\tt \_dispose} on an object implementation, the
pointer to the object should not be used any more. At the time the {\tt
\_dispose} call is made, there may be other threads invoking on the object,
omniORB2 ensures that all these calls are completed before removing the
object from its internal tables and releasing the resources associated with
it. The storage associated with the object is released by omniORB2
using the delete operator. This is why all object implementation should be
initialised using the new operator (section~\ref{objimpl}).

The disposal of an object implementation by omniORB2 may also be deferred
when {\bf colocated} clients continue to hold on to copies of the object's
reference\footnote{Object references held by clients in other address
spaces will not prevent the object implementation from being disposed
of. If these clients invoke on the object after it is disposed, the system
exception INV\_OBJREF is raised.}. This behavior is to prevent the
short-circuited calls from the clients to fail unpredictably.

To summarise, an application can make no assumption as to when the object
is disposed by omniORB2 after the {\tt \_dispose} call returns. If it is
necessary to have better control on when to stop serving incoming
requests, the work should be done by the object implementation itself, such
as by keeping track of the current serving state.

\section{Example 2 - Different Address Spaces}

In this example, the client and the object implementation reside in two
different address spaces. The code of this example is almost the same as the
previous example. The only difference is the extra work need to be done to
pass the object reference from the object implementation to the client.

The simplest (and quite primitive) way to pass an object reference between
two address spaces is to produce a stringified version of the object
reference and to pass this string to the client as a command-line argument.
The string is then converted by the client into a proper object reference.
This method is used in this example. In the next example, we shall
introduce a better way of passing the object reference using the COS Naming
Service.

\subsection{Object Implementation: Generating a Stringified Object Reference}

The {\tt main} function of the object implementation side is reproduced
below. The full listing ({\tt eg2\_impl.cc}) can be found at the end of
this chapter.

{\small
\begin{verbatim}
int
main(int argc, char **argv)
{
  CORBA::ORB_ptr orb = CORBA::ORB_init(argc,argv,"omniORB2");
  CORBA::BOA_ptr boa = orb->BOA_init(argc,argv,"omniORB2_BOA");

  Echo_i *myobj = new Echo_i();
  myobj->_obj_is_ready(boa);

  {
    Echo_var myobjRef = myobj->_this();
    CORBA::String_var p;

    p = orb->object_to_string(myobjRef);            //line 1

    cerr << "'" << (char*)p << "'" << endl;
  }

  boa->impl_is_ready();    // block here indefinitely
                           // See the explanation in example 1
  return 0;
}
\end{verbatim}
}

The stringified object reference is obtained by calling the ORB's function
{\tt \_object\_to\_string} (line 1). This is a sequence starting with the
signature ``IOR:'' and followed by a hexadecimal string. All CORBA 2.0
compliant ORBs are able to convert the string into its internal
representation of a so-called Interoperable Object Reference (IOR). The IOR
contains the location information and a key to uniquely identify the object
implementation in its own address space\footnote{Notice that the
object key is not globally unique across address spaces.}. From the IOR, an
object reference can be constructed.

\subsection{Client: Using a Stringified Object Reference}
\label{clnt2}

The stringified object reference is passed to the client as a command-line
argument. The client uses the ORB's function {\tt string\_to\_object} to
convert the string into a generic object reference ({\tt
CORBA::Object\_ptr}). The relevant section of the code is reproduced
below. The full listing ({\tt eg2\_clt.cc}) can be found at the end of this
chapter.

{\small
\begin{verbatim}
try {
  CORBA::Object_var obj = orb->string_to_object(argv[1]);
  hello(obj);
}
catch(CORBA::COMM_FAILURE& ex) {
  ... // code to handle communication failure
}
\end{verbatim}
}

\subsection{Catching System Exceptions}

When omniORB2 detects an error condition, it may raise a system exception.
The CORBA specification defines a series of exceptions covering most of the
error conditions that an ORB may encounter. The client may choose to catch
these exceptions and recover from the error condition\footnote{If a system
exception is not caught, the C++ runtime will call the {\tt terminate}
function. This function is defaulted to abort the whole process and on some
system will cause a core file to be produced.}. For instance, the code
fragment, shown in section~\ref{clnt2}, catches the system exception
COMM\_FAILURE which indicates that communication with the object
implementation in another address space has failed. 

All system exceptions inherit from the class
{\tt CORBA::SystemException}. With compilers that support RTTI\footnote{Run Time
Type Identification}\footnote{A noticeable exception is the GNU C++
compiler (version 2.7.2). It doesn't support RTTI unless the
compilation flag -frtti is specified. The omniORB2 runtime is not compiled
with the -frtti flag. It is said that RTTI will be properly supported in
the upcoming version 2.8.}, a single catch {\tt CORBA::SystemException} will
catch all the different system exceptions thrown by omniORB2. 

When omniORB2 detects an internal inconsistency that is most likely to be
caused by a bug in the runtime, it raises the exception {\tt
omniORB::fatalException}.  When this exception is raised, it is not
sensible to proceed with any operation that involves the ORB's runtime. It
is best to exit the program immediately. The exception structure carries by
{\tt omniORB::fatalException} contains the exact location (the file name
and the line number) where the exception is raised. You are strongly
encourage to file a bug report and point out the location.

\subsection{Lifetime of an Object Implementation}

It may be obvious but it has to stated that an object implementation exists
only for the duration of the process's lifetime. When the same program is
run again, a different instance of the object implementation is
created. More significantly, {\bf the IOR, and hence the object reference,
of this instance is different from that of the previous run}.

For instance, if you look at the stringified object reference produced by
the program {\tt eg2\_impl} in different runs, they are all
different. The implication is that you cannot store away the stringified
object reference and expect to be able to use it again later when the
original program run has terminated.

For system services and other applications, it may be desirable to have
``persistent'' object implementations.  The objects are ``persistent'' in
the sense that they can be contacted using the same IOR when they are
instantiated in different program runs. To provide this functionality,
omniORB2 needs to be provided with two pieces of information: the (network)
location and the object key. The details of how this can be done will be
described in the later part of this manual.

Alternatively, an indirection from textual pathnames to object references
can be used. Applications can register object implementations at runtime to
a naming service and bind them to fixed pathnames. Clients can bind to the
object implementations at runtime by asking the naming service to resolve
the pathnames to the object references. CORBA defines a naming service,
which is a component of the Common Object Services
(COS)~\cite{corbaservices}, that can be used for this purpose. The next
section describes an example of how to use the COS Naming Service.

\section{Example 3 - Using the COS Naming Service}

In this example, the object implementation uses the COS Naming
Service~\cite{corbaservices} to
pass on the object reference to the client.  This method is by-far more
practical than using stringified object references. The full listing of
the object implementation ({\tt eg3\_impl.cc}) and the client ({\tt
eg3\_clt.cc}) can be found at the end of this chapter.

The object reference is bound to the pathname ``{\bf
test}/{\bf Echo}''\footnote{A pathname, or in the Naming Service's
terminology- a {\it compound name}, is a sequence of textual names. Each
name component except the last one is bound to a naming context. A naming
context is analogous to a directory in a filing system, it can contain
names of object references or other naming contexts. The last name
component is bound to an object reference. Note: '/' is purely a notation
to separate two components in the pathname. It does not appear in the {\it
compound name} that is registered with the Naming Service.}. The pathname
consists of the context {\bf test} and the object name {\bf Echo}. Both the
context and the object name has an attribute {\bf kind}. This attribute is
a string that is intended to be used to describe the name in a
syntax-independent way. The naming service does not interpret, assign, or
manage these values. However both the name and the kind attribute must
match for a name lookup to succeed. In this example, the {\bf kind} values
for {\bf test} and {\bf Echo} are chosen to be ``my\_context'' and
``Object'' respectively. This is an arbitrary choice for there is no
standardised set of kind values.

\subsection{Obtaining the Root Context Object Reference}
\label{resolveinit}

The initial contact with the Naming Service can be established via what we
called the {\bf root} context. The object reference to the root context is
provided by the ORB and can be obtained by calling
{\tt resolve\_initial\_references}. The following code fragment shows how
it is used:

{\small
\begin{verbatim}

CORBA::ORB_ptr orb = CORBA::ORB_init(argc,argv,"omniORB2");

CORBA::Object_var initServ;
initServ = orb->resolve_initial_references("NameService");

CosNaming::NamingContext_var rootContext;
rootContext = CosNaming::NamingContext::_narrow(initServ);

\end{verbatim}
}

Remember, omniORB2 constructs its internal list of initial references at
initialisation time using the information provided in the configuration
file {\tt omniORB.cfg}. If this file is not present, the internal list will
be empty and {\tt resolve\_initial\_references} will raise a 
CORBA::ORB::InvalidName exception. 

\subsection{The Naming Service Interface}

It is beyond the scope of this chapter to describe in detail the Naming
Service interface. You should consult the CORBAservices
specification~\cite{corbaservices} (chapter 3). The
code listed in {\tt eg3\_impl.cc} and {\tt eg3\_clt.cc} are good examples
of how the service can be used. Please spend time to study the examples
carefully.

\newpage
\section{Source Listing}

{\small
\subsection{echo\_i.cc}
\begin{verbatim}
// echo_i.cc - This source code demonstrates an implmentation of the
//             object interface Echo. It is part of the three examples
//             used in Chapter 2 "The Basics" of the omniORB2 user guide.
//
#include <string.h>
#include "echo.hh"

class Echo_i : public virtual _sk_Echo {
public:
  Echo_i() {}
  virtual ~Echo_i() {}
  virtual char * echoString(const char *mesg);
};

char *
Echo_i::echoString(const char *mesg) {
  char *p = CORBA::string_dup(mesg);
  return p;
}
\end{verbatim}
\newpage

\subsection{greeting.cc}
\begin{verbatim}
// greeting.cc - This source code demonstrates the use of an object
//               reference by a client to perform an operation on an 
//               object. It is part of the three examples used
//               in Chapter 2 "The Basics" of the omniORB2 user guide.
//
#include <iostream.h>
#include "echo.hh"

void
hello(CORBA::Object_ptr obj)
{
  Echo_var e = Echo::_narrow(obj);

  if (CORBA::is_nil(e)) {
    cerr << "hello: cannot invoke on a nil object reference.\n" << endl;
    return;
  }

  CORBA::String_var src = (const char*) "Hello!";  
  CORBA::String_var dest;

  dest = e->echoString(src);

  cerr << "I said,\"" << src << "\"."
       << " The Object said,\"" << dest <<"\"" << endl;
}
\end{verbatim}
\newpage

\subsection{eg1.cc}
\begin{verbatim}
// eg1.cc - This is the source code of example 1 used in Chapter 2 
//          "The Basics" of the omniORB2 user guide.
//
//          In this example, both the object implementation and the 
//          client are in the same process.
//
// Usage: eg1
//
#include <iostream.h>
#include "echo.hh"

#include "echo_i.cc"
#include "greeting.cc"

int
main(int argc, char **argv)
{
  CORBA::ORB_ptr orb = CORBA::ORB_init(argc,argv,"omniORB2");
  CORBA::BOA_ptr boa = orb->BOA_init(argc,argv,"omniORB2_BOA");

  Echo_i *myobj = new Echo_i();
  // Note: all implementation objects must be instantiated on the
  // heap using the new operator.

  myobj->_obj_is_ready(boa);
  // Tell the BOA the object is ready to serve.
  // This call is omniORB2 specific.
  //
  // This call is equivalent to the following call sequence:
  //     Echo_ptr myobjRef = myobj->_this();
  //     boa->obj_is_ready(myobjRef);
  //     CORBA::release(myobjRef);

  boa->impl_is_ready(0,1);
  // Tell the BOA we are ready and to return immediately once it has
  // done its stuff. It is omniORB2 specific to call impl_is_ready()
  // with the extra 2nd argument- CORBA::Boolean NonBlocking,
  // which is set to TRUE (1) in this case.

  Echo_ptr myobjRef = myobj->_this();
  // Obtain an object reference.
  // Note: always use _this() to obtain an object reference from the
  //       object implementation.

  hello(myobjRef);

  CORBA::release(myobjRef);
  // Dispose of the object reference.

  myobj->_dispose();
  // Dispose of the object implementation.
  // This call is omniORB2 specific.
  // Note: *never* call the delete operator or the dtor of the object
  //       directly because the BOA needs to be informed.
  //
  // This call is equivalent to the following call sequence:
  //     Echo_ptr myobjRef = myobj->_this();
  //     boa->dispose(myobjRef);
  //     CORBA::release(myobjRef);

  return 0;
}
\end{verbatim}
\newpage

\subsection{eg2\_impl.cc}
\begin{verbatim}
// eg2_impl.cc - This is the source code of example 2 used in Chapter 2
//               "The Basics" of the omniORB2 user guide.
//
//               This is the object implementation.
//
// Usage: eg2_impl
//
//        On startup, the object reference is printed to cerr as a
//        stringified IOR. This string should be used as the argument to 
//        eg2_clt.
//
#include <iostream.h>
#include "omnithread.h"
#include "echo.hh"

#include "echo_i.cc"

int
main(int argc, char **argv)
{
  CORBA::ORB_ptr orb = CORBA::ORB_init(argc,argv,"omniORB2");
  CORBA::BOA_ptr boa = orb->BOA_init(argc,argv,"omniORB2_BOA");

  Echo_i *myobj = new Echo_i();
  myobj->_obj_is_ready(boa);

  {
    Echo_var myobjRef = myobj->_this();
    CORBA::String_var p = orb->object_to_string(myobjRef);
    cerr << "'" << (char*)p << "'" << endl;
  }

  boa->impl_is_ready();
  // Tell the BOA we are ready. The BOA's default behaviour is to block
  // on this call indefinitely.

  return 0;
}
\end{verbatim}
\newpage

\subsection{eg2\_clt.cc}
\begin{verbatim}
// eg2_clt.cc - This is the source code of example 2 used in Chapter 2
//              "The Basics" of the omniORB2 user guide.
//
//              This is the client. The object reference is given as a
//              stringified IOR on the command line.
//
// Usage: eg2_clt <object reference>
//
#include <iostream.h>
#include "echo.hh"

#include "greeting.cc"

extern void hello(CORBA::Object_ptr obj);

int
main (int argc, char **argv) 
{
  CORBA::ORB_ptr orb = CORBA::ORB_init(argc,argv,"omniORB2");
  CORBA::BOA_ptr boa = orb->BOA_init(argc,argv,"omniORB2_BOA");

  if (argc < 2) {
    cerr << "usage: eg2_clt <object reference>" << endl;
    return 1;
  }

  try {
    CORBA::Object_var obj = orb->string_to_object(argv[1]);
    hello(obj);
  }
  catch(CORBA::COMM_FAILURE& ex) {
    cerr << "Caught system exception COMM_FAILURE, unable to contact the "
         << "object." << endl;
  }
  catch(omniORB::fatalException& ex) {
    cerr << "Caught omniORB2 fatalException. This indicates a bug is caught "
         << "within omniORB2.\nPlease send a bug report.\n"
         << "The exception was thrown in file: " << ex.file() << "\n"
         << "                            line: " << ex.line() << "\n"
         << "The error message is: " << ex.errmsg() << endl;
  }
  catch(...) {
    cerr << "Caught a system exception." << endl;
  }

  return 0;
}
\end{verbatim}
\newpage

\subsection{eg3\_impl.cc}
\begin{verbatim}
// eg3_impl.cc - This is the source code of example 3 used in Chapter 2
//               "The Basics" of the omniORB2 user guide.
//
//               This is the object implementation.
//
// Usage: eg3_impl
//
//        On startup, the object reference is registered with the 
//        COS naming service. The client uses the naming service to
//        locate this object.
//
//        The name which the object is bound to is as follows:
//              root  [context]
//               |
//              text  [context] kind [my_context]
//               |
//              Echo  [object]  kind [Object]
//

#include <iostream.h>
#include "omnithread.h"
#include "echo.hh"

#include "echo_i.cc"

static CORBA::Boolean bindObjectToName(CORBA::ORB_ptr,CORBA::Object_ptr);

int
main(int argc, char **argv)
{
  CORBA::ORB_ptr orb = CORBA::ORB_init(argc,argv,"omniORB2");
  CORBA::BOA_ptr boa = orb->BOA_init(argc,argv,"omniORB2_BOA");

  Echo_i *myobj = new Echo_i();
  myobj->_obj_is_ready(boa);

  {
    Echo_var myobjRef = myobj->_this();
    if (!bindObjectToName(orb,myobjRef)) {
      return 1;
    }
  }

  boa->impl_is_ready();
  // Tell the BOA we are ready. The BOA's default behaviour is to block
  // on this call indefinitely.

  return 0;
}


static
CORBA::Boolean
bindObjectToName(CORBA::ORB_ptr orb,CORBA::Object_ptr obj)
{
  CosNaming::NamingContext_var rootContext;
  
  try {
    // Obtain a reference to the root context of the Name service:
    CORBA::Object_var initServ;
    initServ = orb->resolve_initial_references("NameService");

    // Narrow the object returned by resolve_initial_references()
    // to a CosNaming::NamingContext object:
    rootContext = CosNaming::NamingContext::_narrow(initServ);
    if (CORBA::is_nil(rootContext))
      {
        cerr << "Failed to narrow naming context." << endl;
        return 0;
      }
  }
  catch(CORBA::ORB::InvalidName& ex) {
    cerr << "Service required is invalid [does not exist]." << endl;
    return 0;
  }


  try {
    // Bind a context called "test" to the root context:

    CosNaming::Name contextName;
    contextName.length(1);
    contextName[0].id   = (const char*) "test";    // string copied
    contextName[0].kind = (const char*) "my_context"; // string copied    
    // Note on kind: The kind field is used to indicate the type
    // of the object. This is to avoid conventions such as that used
    // by files (name.type -- e.g. test.ps = postscript etc.)

    CosNaming::NamingContext_var testContext;
    try {
      // Bind the context to root, and assign testContext to it:
      testContext = rootContext->bind_new_context(contextName);
    }
    catch(CosNaming::NamingContext::AlreadyBound& ex) {
      // If the context already exists, this exception will be raised.
      // In this case, just resolve the name and assign testContext
      // to the object returned:
      CORBA::Object_var tmpobj;
      tmpobj = rootContext->resolve(contextName);
      testContext = CosNaming::NamingContext::_narrow(tmpobj);
      if (CORBA::is_nil(testContext)) {
        cerr << "Failed to narrow naming context." << endl;
        return 0;
      }
    } 

    // Bind the object (obj) to testContext, naming it Echo:
    CosNaming::Name objectName;
    objectName.length(1);
    objectName[0].id   = (const char*) "Echo";   // string copied
    objectName[0].kind = (const char*) "Object"; // string copied


    // Bind obj with name Echo to the testContext:
    try {
      testContext->bind(objectName,obj);
    }
    catch(CosNaming::NamingContext::AlreadyBound& ex) {
      testContext->rebind(objectName,obj);
    }
    // Note: Using rebind() will overwrite any Object previously bound 
    //       to /test/Echo with obj.
    //       Alternatively, bind() can be used, which will raise a
    //       CosNaming::NamingContext::AlreadyBound exception if the name
    //       supplied is already bound to an object.

    // Amendment: When using OrbixNames, it is necessary to first try bind
    // and then rebind, as rebind on it's own will throw a NotFoundexception if
    // the Name has not already been bound. [This is incorrect behaviour -
    // it should just bind].
  }
  catch (CORBA::COMM_FAILURE& ex) {
    cerr << "Caught system exception COMM_FAILURE, unable to contact the "
         << "naming service." << endl;
    return 0;
  }
  catch (omniORB::fatalException& ex) {
    throw;
  }
  catch (...) {
    cerr << "Caught a system exception while using the naming service."<< endl;
    return 0;
  }
  return 1;
}
\end{verbatim}
\newpage

\subsection{eg3\_clt.cc}
\begin{verbatim}
// eg3_clt.cc - This is the source code of example 3 used in Chapter 2
//              "The Basics" of the omniORB2 user guide.
//
//              This is the client. It uses the COSS naming service
//              to obtain the object reference.
//
// Usage: eg3_clt
//
//
//        On startup, the client lookup the object reference from the
//        COS naming service.
//
//        The name which the object is bound to is as follows:
//              root  [context]
//               |
//              text  [context] kind [my_context]
//               |
//              Echo  [object]  kind [Object]
//

#include <iostream.h>
#include "echo.hh"

#include "greeting.cc"

extern void hello(CORBA::Object_ptr obj);

static CORBA::Object_ptr getObjectReference(CORBA::ORB_ptr orb);

int
main (int argc, char **argv) 
{
  CORBA::ORB_ptr orb = CORBA::ORB_init(argc,argv,"omniORB2");
  CORBA::BOA_ptr boa = orb->BOA_init(argc,argv,"omniORB2_BOA");

  try {
    CORBA::Object_var obj = getObjectReference(orb);
    hello(obj);
  }
  catch(CORBA::COMM_FAILURE& ex) {
    cerr << "Caught system exception COMM_FAILURE, unable to contact the "
         << "object." << endl;
  }
  catch(omniORB::fatalException& ex) {
    cerr << "Caught omniORB2 fatalException. This indicates a bug is caught "
         << "within omniORB2.\nPlease send a bug report.\n"
         << "The exception was thrown in file: " << ex.file() << "\n"
         << "                            line: " << ex.line() << "\n"
         << "The error message is: " << ex.errmsg() << endl;
  }
  catch(...) {
    cerr << "Caught a system exception." << endl;
  }

  return 0;
}

static 
CORBA::Object_ptr
getObjectReference(CORBA::ORB_ptr orb)
{
  CosNaming::NamingContext_var rootContext;
  
  try {
    // Obtain a reference to the root context of the Name service:
    CORBA::Object_var initServ;
    initServ = orb->resolve_initial_references("NameService");

    // Narrow the object returned by resolve_initial_references()
    // to a CosNaming::NamingContext object:
    rootContext = CosNaming::NamingContext::_narrow(initServ);
    if (CORBA::is_nil(rootContext)) 
      {
        cerr << "Failed to narrow naming context." << endl;
        return CORBA::Object::_nil();
      }
  }
  catch(CORBA::ORB::InvalidName& ex) {
    cerr << "Service required is invalid [does not exist]." << endl;
    return CORBA::Object::_nil();
  }


  // Create a name object, containing the name test/context:
  CosNaming::Name name;
  name.length(2);

  name[0].id   = (const char*) "test";       // string copied
  name[0].kind = (const char*) "my_context"; // string copied
  name[1].id   = (const char*) "Echo";
  name[1].kind = (const char*) "Object";
  // Note on kind: The kind field is used to indicate the type
  // of the object. This is to avoid conventions such as that used
  // by files (name.type -- e.g. test.ps = postscript etc.)

  
  CORBA::Object_ptr obj;
  try {
    // Resolve the name to an object reference, and assign the reference 
    // returned to a CORBA::Object:
    obj = rootContext->resolve(name);
  }
  catch(CosNaming::NamingContext::NotFound& ex)
    {
      // This exception is thrown if any of the components of the
      // path [contexts or the object] aren't found:
      cerr << "Context not found." << endl;
      return CORBA::Object::_nil();
    }
  catch (CORBA::COMM_FAILURE& ex) {
    cerr << "Caught system exception COMM_FAILURE, unable to contact the "
         << "naming service." << endl;
    return CORBA::Object::_nil();
  }
  catch(omniORB::fatalException& ex) {
    throw;
  }
  catch (...) {
    cerr << "Caught a system exception while using the naming service."<< endl;
    return CORBA::Object::_nil();
  }
  return obj;
}
\end{verbatim}
\newpage


\newpage
\subsection{dir.mk}
\begin{verbatim}
# dir.mk - This is the makefile to compile all the example programs used in
#          Chapter 2 The Basics.
#          This makefile is to be used under the OMNI development environment.
#
CXXSRCS = greeting.cc eg1.cc \
          eg2_impl.cc eg2_clt.cc \
          eg3_impl.cc eg3_clt.cc

DIR_CPPFLAGS = $(CORBA_CPPFLAGS)

CORBA_INTERFACES = echo

eg1        = $(patsubst %,$(BinPattern),eg1)
eg2_impl   = $(patsubst %,$(BinPattern),eg2_impl)
eg2_clt    = $(patsubst %,$(BinPattern),eg2_clt)
eg3_impl   = $(patsubst %,$(BinPattern),eg3_impl)
eg3_clt    = $(patsubst %,$(BinPattern),eg3_clt)

all:: $(eg1) $(eg2_impl) $(eg2_clt)  $(eg3_impl) $(eg3_clt)

clean::
        $(RM) $(eg1) $(eg2_impl) $(eg2_clt) $(eg3_impl) $(eg3_clt)

$(eg1): eg1.o $(CORBA_STUB_OBJS) $(CORBA_LIB_DEPEND)
        @(libs="$(CORBA_LIB)"; $(CXXExecutable))

$(eg2_impl): eg2_impl.o $(CORBA_STUB_OBJS) $(CORBA_LIB_DEPEND)
        @(libs="$(CORBA_LIB)"; $(CXXExecutable))

$(eg2_clt): eg2_clt.o $(CORBA_STUB_OBJS) $(CORBA_LIB_DEPEND)
        @(libs="$(CORBA_LIB)"; $(CXXExecutable))

$(eg3_impl): eg3_impl.o $(CORBA_STUB_OBJS) $(CORBA_LIB_DEPEND)
        @(libs="$(CORBA_LIB)"; $(CXXExecutable))

$(eg3_clt): eg3_clt.o $(CORBA_STUB_OBJS) $(CORBA_LIB_DEPEND)
        @(libs="$(CORBA_LIB)"; $(CXXExecutable))

\end{verbatim}
}

%\input{api.tex}

\chapter{The omniORB2 API}

In this chapter, we introduce the omniORB2 API. The purpose
of this API is to provide access points to omniORB2 specific
functionalities that are not covered by the CORBA specification.
Obviously, if you use this API in your application, that part of your code
is not going to be portable to run unchanged on other vendors' ORBs. To
make it easier to identify omniORB2 dependent code, this API is defined
under the name space ``omniORB''\footnote{omniORB is a class name if the
C++ compiler does not support the namespace keyword.}.

\section{ORB and BOA initialization options}

{\tt CORBA::ORB\_init} accepts the following command-line arguments:

\begin{description}

\item[-ORBid ``omniORB2''] The identifier supplied must be ``omniORB2''.
\item[-ORBtraceLevel $<$level$>$] This option is described in section~\ref{rttrace}.

\end{description}

{\tt BOA\_init} accepts the following command-line arguments:

\begin{description}

\item[-BOAid ``omniORB2\_BOA''] The identifier supplied must be ``omniORB2\_BOA''.
\item[-BOAiiop\_port $<$port number$>$] This option tells the BOA which
TCP/IP port to use to accept IIOP calls. If this option is not specified,
the BOA will use an arbitrary port assigned by the operating system.


\end{description}


As defined in the CORBA specification, any command-line arguments
understood by the ORB/BOA will be removed from {\tt argv} when the
initialisation functions return. Therefore, an application is not required
to handle any command-line arguments it does not understand.

\section{Run-time Tracing and Diagnostic Messages}
\label{rttrace}

OmniORB2 uses the C++ iostream {\tt cerr} to output any tracing and
diagnostic messages. Some or all of these messages can be turned-{on/off} by
setting the variable {\tt omniORB::traceLevel}. The type definition of
the variable is:

{\small
\begin{verbatim}
CORBA::ULong omniORB::traceLevel = 1;  // The default value is 1
\end{verbatim}
}

At the moment, the following trace levels are defined:

\begin{description}

\item[level 0] turn off all tracing and informational messages
\item[level 1] informational messages only
\item[level 2] the above plus configuration information
\item[level 5] the above plus notifications when server threads are created
or communication endpoints are shutdown
\item[level 10] the above plus execution traces

\end{description}

The variable can be changed by assignment inside your applications. It can
also be changed by specifying the command-line option: {\tt -ORBtraceLevel
$<$level$>$}. For instance:

{\small
\begin{verbatim}
$ eg2_impl -ORBtraceLevel 5
\end{verbatim}
}

%$

\section{Object Keys}

OmniORB2 uses a data type {\tt omniORB::objectKey} to uniquely
identify each object implementation. This is an opaque data type and 
can only be manipulated by the following functions:

{\small
\begin{verbatim}
void omniORB::generateNewKey(omniORB::objectKey &k);
\end{verbatim}
}

{\tt omniORB::generateNewKey} returns a new {\tt objectKey}. The return
value is guaranteed to be unique among the keys generated during this program
run. On the platforms that have a realtime clock and unique process
identifiers, a stronger assertion can be made, i.e. the keys are guaranteed
to be unique among all keys ever generated on the same machine.

{\small
\begin{verbatim}
const unsigned int omniORB::hash_table_size;
int omniORB::hash(omniORB::objectKey& k);
\end{verbatim}
}

{\tt omniORB::hash} returns the hash value of an {\tt objectKey}. The value
returned by this function is always between 0 and {\tt
omniORB:hash\_table\_size - 1} inclusively.

{\small
\begin{verbatim}
omniORB::objectKey omniORB::nullkey();
\end{verbatim}
}

{\tt omniORB::nullkey} always returns the same {\tt objectKey} value. This
key is guaranteed to hash to 0.

{\small
\begin{verbatim}
int operator==(const omniORB::objectKey &k1,const omniORB::objectKey &k2);
int operator!=(const omniORB::objectKey &k1,const omniORB::objectKey &k2);
\end{verbatim}
}

{\tt ObjectKeys} can be tested for equality using the overloaded {\tt
operator==} and {\tt operator!=}.

{\small
\begin{verbatim}
omniORB::seqOctets*
omniORB::keyToOctetSequence(const omniORB::objectKey &k1);

omniORB::objectKey
omniORB::octetSequenceToKey(const omniORB::seqOctets& seq);
\end{verbatim}
}

{\tt omniORB::keyToOctetSequence} takes an {\tt objectKey} and returns its
externalised representation in the form of a sequence of octets. The same
sequence can be converted back to an {\tt objectKey} using {\tt
omniORB::octetSequenceToKey}. If the supplied sequence is not an {\tt
objectKey}, {\tt omniORB::octetSequenceToKey} raises a {\tt CORBA::MARSHAL}
exception.

\section{Trapping omniORB2 Internal Errors}

{\small
\begin{verbatim}
class fatalException {
public:
    const char *file() const;
    int line() const;
    const char *errmsg() const;
};
\end{verbatim}
}

When omniORB2 detects an internal inconsistency that is most likely to be
caused by a bug in the runtime, it raises the exception {\tt
omniORB::fatalException}.  When this exception is raised, it is not
sensible to proceed with any operation that involves the ORB's runtime. It
is best to exit the program immediately. The exception structure carries by
{\tt omniORB::fatalException} contains the exact location (the file name
and the line number) where the exception is raised. You are strongly
encourage to file a bug report and point out the location.



%\chapter{Mapping for Interfaces}

%\centerline{\it [Empty]}

%\chapter{Multi-Threading}

%\chapter{Exceptions}

%\chapter{IDL to C++ Language Mapping}

%\chapter{Object Activation}

%\chapter{Dynamic Interfaces}

\begin{thebibliography}{tjr96}

\bibitem[OMG96a]{corba2-spec} 
{\em The Common Object Request Broker: Architecture and
  Specification}, 
Revision 2.0, 
OMG,
Updated July 1996.

\bibitem[OMG96b]{corbaservices} 
{\em CORBAservices: Common Object Services Specification},
OMG,
Updated July 1996.

\bibitem[Richardson96a]{tjr96a}
{\em The OMNI Thread Abstraction},
Tristan Richardson, ORL, 22 October 1996.

\bibitem[Richardson96b]{tjr96b}
{\em The OMNI Development Environment Version 4.0},
Tristan Richardson, ORL, 5 November 1996.

\end{thebibliography}

\end{document}
