\documentclass[11pt]{article}
\usepackage{a4}
\usepackage{fancyheadings}
\pagestyle{fancyplain}
\parskip 1ex
\parindent 0pt
\title{The OMNI Development Environment Version 4.0}
\author{Tristan Richardson\\
        ORL\\
        Cambridge}
\date{15 October 1996}

\begin{document}

\maketitle

{\Large Quick Summary of Differences from the Current ODE}

\begin{itemize}

\item uses GNU make (no imake!) on all platforms.

\item rather than having an ``install tree'', it allows multiple {\em
import trees} and a single {\em export tree}.

\item for multiple {\em build trees} from a common source, symbolic links are
no longer needed.  Instead it uses GNU make's VPATH feature to search for
source files in an arbitrary number of {\em source trees}.

\item this means that build trees are just the same on NT as on unix.  On NT it
uses Cygnus's GNU-WIN32 environment for {\em make}, but all compilation and
linking is still done using MS Visual C++ tools.

\end{itemize}

\section{Introduction}

The OMNI Development Environment (ODE) provides a mechanism for building
software across all our diverse platforms.

The idea is that you create a {\em build tree} for a particular platform where
you can ``build'' software using compilers, linkers, etc.  To help with this
you can pull in things like header files and libraries from a number of {\em
import trees}, and possibly then {\em export} the resulting code to an {\em
export tree} from where others may {\em import} it.  A few simple definitions
of the different types of tree may help:

\begin{itemize}

\item An {\em import/export tree} is a place for sharing compiled libraries and
binaries, header files, IDL files, {\em make} variables and rules.  It has a
standard structure of subdirectories - at present {\tt include}, {\tt idl},
{\tt mk}, {\tt bin} and {\tt lib} - but a particular tree might have some of
these missing or have other extra subdirectories.  Underneath {\tt bin} and
{\tt lib} are platform-dependent subdirectories.

\item A {\em build tree} is where source code is ``built''.  The source code
can be local to the build tree, or it can be pulled in from a source tree which
is shared with other build trees.  The results of building, such as libraries
and binaries, can be exported to an {\em export tree}.  Such an {\em export
tree} can then be used as an {\em import tree} by another build tree.
Obviously build trees are platform-dependent.

\item A {\em source tree} has the source code of things which need ``building''
such as C/C++ source and java source.  Things which don't need to be ``built''
such as header files and IDL files can just be put directly into an
import/export tree.

\end{itemize}

By convention a source tree will often be associated with an import/export
tree and will be placed in the {\tt src} subdirectory of the import/export
tree.  Its structure underneath {\tt src} can be anything.

For each platform there will also often be a build tree under the {\tt build}
subdirectory of the import/export tree.  Here source code from the
corresponding source tree is built, and then exported to the import/export
tree.

Note that when you treat an import/export tree as an import tree, the contents
of its corresponding source and build trees is totally irrelevant.  There is no
sense in which you can ``import'' from a build tree.  This differs from the old
OMNI development environment where, for example, an {\tt appl} build tree could
``import'' from a {\tt lib} build tree.

Every build tree must have a {\em base} OMNI tree from which it imports.
Usually this will be the current {\em release} or {\em develop} OMNI tree.  In
addition the build tree can have any number of other import trees.  The order
of these import trees is important - if a library, header file or whatever
appears in more than one import tree, the build tree will use the one in the
import tree which is specified first.  The {\em base} OMNI tree will always be
searched last.


\section{Some example ways of using the new ODE}

There are a number of different ways in which the ODE can be used.  Here are
some examples, using various source, build and import/export trees.  Don't
worry if the later examples seem complicated - it will all become clear when
you try some building yourself.

\begin{enumerate}

\item
\label{buildtreeonly}
The simplest situation is where a user wants to write their own code in
isolation.  For a single platform this can be done directly in a build tree
which just imports from the base OMNI tree but has no export tree or source
trees.  See the hello world example below.

\item If the user wants to build the same source code on multiple platforms the
code can be placed in a source tree.  Several build trees can then be created
each of which shadow the source tree (i.e. have the same directory structure
but no files).  Again these build trees need not have an export tree.

\item
\label{twoimporttrees}
To share header files and libraries between several programs, the user
needs their own import/export tree.  The build tree(s) should both import from
and export to this import/export tree, as well as importing from the base OMNI
tree.  Note that if two programs want to share a library which is built in the
same tree, the recommended way of doing this is to export the library to an
import/export tree, rather than just pulling the library out of the build tree.
This makes it possible to build the same programs in a different tree simply by
importing from this import/export tree.

\item The base OMNI tree has a source tree and a build tree for each platform.
Each of these build trees shadows the source tree, and both imports from and
exports to the base OMNI import/export tree.  It has no other import trees.

\item If the user wants to build their own version of a standard OMNI library
with just one or two source file changes, this can be done by using the base
OMNI tree as a second {\em source tree}.  The user makes a directory with
exactly the same path in their own source tree and build trees as in the base
OMNI source tree.  A local file in the user's source tree then just overrides
the same file in the base OMNI source tree.  Once the library has been built,
it should be exported to the user's import/export tree, where it can be used to
override the library in the base OMNI import/export tree.

\item Each project using the ODE will have its own import/export tree with
corresponding source tree and build trees for each platform.  Similar to
case~\ref{twoimporttrees} these build trees will import from and export to the
project's import/export tree and also import from the base OMNI tree.

\item People working on a project will also have personal build trees which
import from the project tree followed by the base OMNI tree.  They may also
have their own import/export tree which is imported from first.  In this case
there are three levels of import tree.  There is theoretically no limit to the
number of trees which may be imported in this way.

\item Similarly to multiple import trees, a user may have a build tree with
multiple source trees.  It could firstly shadow their own source tree followed
by a project source tree followed by the base OMNI source tree.  This could be
used to build a different version of a standard OMNI library, a different
version of a standard project library and the user's own programs all in the
same build tree.

\end{enumerate}

\section{Writing a Hello World program}

This section introduces the use of the development environment by going through
the steps to build a simple ``hello world'' program on a unix platform.  This
will be done using the ODE in the simplest way as in case~\ref{buildtreeonly}
above.

\subsection{Setting up Your Environment}

Although not strictly necessary it is advisable to set the environment variable
{\tt OMNI\_TREE} to point to the base OMNI tree you wish to use (for example
{\tt /project/omni/version4.0}).  Then add the directories {\tt
\$OMNI\_TREE/bin/}{\it platform} and {\tt \$OMNI\_TREE/bin/scripts} to your
path in that order (so that platform-specific binaries will override scripts).
You may also wish to add {\tt \$OMNI\_TREE/man} to your {\tt MANPATH}.  If the
version of make which is on your path is not GNU make then you need to set the
environment variable {\tt GNUMAKE} to specify the command to run for GNU make.

\subsection{Creating a build tree}

You create a build tree by running the script {\tt obuildtree}.  It will prompt
you for a number of things.  Often there is a sensible default shown in square
brackets which will be used if you just hit RETURN.  In this case you want a
build tree for whichever platform you're on with no source or export tree.  You
can name the destination directory whatever you like.  For example:

\begin{verbatim}
$ obuildtree

Enter the base OMNI tree for this build tree
> [/project/omni/version4.0] 

Enter the platform name from:
alpha_osf1_3.2
arm_atmos_3.0/atb
i86_nt_3.5
sun4_sosV_5.5
> sun4_sosV_5.5

Enter the destination directory
> [sun4_sosV_5.5] solaris_hello

Enter the import trees in search order (the base OMNI tree
"/project/omni/version4.0" will always be searched after these)
> 

Does this build tree have an export tree ?
> [y] n

Does this build tree have any source trees ?
> [y] n

Creating OMNI build tree "solaris_hello" for platform "sun4_sosV_5.5"

$
\end{verbatim}

Now go into your new build tree:

\begin{verbatim}
$ cd solaris_hello
\end{verbatim}

Notice that there is a directory called {\tt config} with files called {\tt
config.mk} and {\tt sources}.  Generating these files is all that {\tt
obuildtree} does.

Now you're ready to write your ``hello world'' program:

\begin{verbatim}
$ cat >hello.c
#include <stdio.h>
int main()
{
  printf("Hello world!\n");
  return 0;
}
$
\end{verbatim}

Now you need to specify how to make an executable from this C source file.
This is done in a file called {\tt dir.mk}.  This is very similar in concept to
an Imakefile except that there is no separate preprocessing stage -- {\tt
dir.mk} is actually processed by {\tt make}.  However, to allow it to be a
platform-independent specification, {\tt dir.mk} is not a complete
makefile in itself.  Instead a front-end to {\tt make} is used called {\tt
omake}.  What {\tt omake} does is to find the {\tt config} directory in your
build tree and run {\tt make} on {\tt config.mk}.  This allows for import trees
to define useful {\tt make} variables and rules which can then be used in your
{\tt dir.mk}.

So for our example you need a {\tt dir.mk} like this:

\begin{verbatim}
$ cat >dir.mk
SRCS = hello.c

all:: hello

hello: hello.o
        @$(CExecutable)
$
\end{verbatim}

Notice this looks just like a makefile -- except that the rule {\tt
\$(CExecutable)} has been predefined for you so you don't need to worry about
compilers, linkers and their flags.  Note also that it is important to use a
TAB rather than spaces in the make rule.  The ``@'' is there to tell make not
to generate lots of irrelevant output.

Now you can run ``{\tt omake}'' to build the executable:

{\small
\begin{verbatim}
$ omake

make -r -f ./config/config.mk VPATH= TOP=. CURRENT=. MAKEARG1=

/project/omni/version4.0/mk/afterdir.mk:26: hello.d: No such file or directory
/bin/sh -ec "gcc -M -O   -I.  -I/project/omni/version4.0/include -D__sunos__ -D
__sparc__ -D__OSVERSION__=5 -DSVR4 hello.c | sed 's/hello\\.o/& hello.d/g' > he
llo.d"
gcc -fpcc-struct-return -c -O   -I.  -I/project/omni/version4.0/include -D__sun
os__ -D__sparc__ -D__OSVERSION__=5 -DSVR4 -o hello.o hello.c
+ rm -f hello 
+ gcc -fpcc-struct-return -o hello -L/project/omni/version4.0/lib/sun4_sosV_5.5
 hello.o 
$
\end{verbatim}
}

The result should be that three files are generated.  The file called
{\tt hello} is the executable which you should be able to run:

\begin{verbatim}
$ ./hello
Hello world!
\end{verbatim}

There should also be the object file {\tt hello.o}.  The third file is called
{\tt hello.d}.  This contains the header file dependencies of {\tt hello.c}.
The generation of this file is done automatically whenever {\tt hello.c}
changes -- there is no need for a separate ``make depend'' step.  Any C source
file specified in {\tt SRCS} or C++ file specified in {\tt CXXSRCS} will have
this dependency checking done automatically.

\subsection{Making it portable to NT and ATMos}

Because different platforms adopt different naming conventions for executables,
the {\tt dir.mk} used above is not suitable for use on non-unix platforms.  To
make it truly portable {\tt dir.mk} should look like this:

\begin{verbatim}
SRCS = hello.c

HELLO = $(patsubst %,$(BinPattern),hello)

all:: $(HELLO)

$(HELLO): hello.o
        @$(CExecutable)

clean::
        $(RM) $(HELLO)
\end{verbatim}

This uses GNU make's {\tt patsubst} feature together with a make variable {\tt
BinPattern} to produce the name of the executable file.  For example on NT,
{\tt BinPattern} is set to {\tt \%.exe}, resulting in the executable name {\tt
hello.exe}.  You may recall this is similar to the old OMNI development
environment's {\tt ProgramTargetName} macro.  For completeness there should
also be a rule for removing the executable when ``{\tt omake clean}'' is done.

\section{Using a source tree}

Assume now you want to build the same code on several platforms.  Put {\tt
hello.c} and {\tt dir.mk} in a directory called {\tt src}.  Now create a
directory alongside it called {\tt build} and inside there run ``{\tt
obuildtree}''.  This time specify the full path of the {\tt src} directory (or
if you prefer ``{\tt ../../src}'') as a source tree.  You should now be able to
type ``{\tt omake}'' inside the build tree and have it compile your program
even though the source file is not in the local build directory.

You will see that on the command line {\tt omake} gives to GNU {\tt make} it
specifies a {\tt -I} flag causing make to search for the {\tt dir.mk} in the
{\tt src} directory, and sets the VPATH variable so that make searches for
source files in that directory.  In this way there is no need for symbolic
links from the build tree back to the source tree.

You should also now be able to create a similar build tree for another platform
using the same source tree, thereby compiling for both platforms from the same
source files.

\section{Creating an import/export tree}

There is no magic involved in creating an import/export tree.  An empty
directory is a completely valid import/export tree, albeit a not very useful
one.  In the conventional case where there are source and build trees
associated with an import/export tree, the import/export tree will be the
directory above the {\tt src} and {\tt build} directories.

If say you want to put a header file in your import/export tree, simply create
an {\tt include} subdirectory and put the header file in it.  Similarly idl
files can simply be put in an {\tt idl} subdirectory.  However, if you want to
put something which needs ``building'' (such as a library or executable binary)
into an import/export tree, this needs to be done by running ``{\tt omake
export}'' inside a build tree.

Suppose you want to export the {\tt hello} executable to your import/export
tree.  When {\tt obuildtree} asks you for any import trees, specify the full
path of your import/export tree.  When it also asks if the build tree has an
export tree you should type ``y'' and again specify the full path of the
import/export tree (with any luck this will be the default).  Now in your {\tt
dir.mk} you need a rule which says how to export the {\tt hello} executable.
Add the following:

\begin{verbatim}
export:: $(HELLO)
        @$(ExportExecutable)
\end{verbatim}

When you run ``{\tt omake export}'' inside your build tree it should create a
{\tt bin/}{\it platform} directory in your import/export tree and put the hello
executable there.


\section{Writing dir.mk}

The {\tt dir.mk} file is basically just a makefile, so you can put normal make
rules and variable definitions there.  However to make full use of the ODE you
will want to use the predefined variables and rules to make your {\tt dir.mk}
platform-independent.  As with imakefiles, usually the best thing to do is find
an existing dir.mk which does nearly the same thing that you want and copy it.

\subsection{Generating library and executable names}

Since each platform has its own naming conventions for libraries and executable
binaries, you should use the GNU make ``patsubst'' function together with {\tt
LibPattern} and {\tt BinPattern} to generate the appropriate file name.
Assigning a make variable to each library or executable name is probably a good
idea, e.g.:

\begin{verbatim}
lib = $(patsubst %,$(LibPattern),my_library)
prog1 = $(patsubst %,$(BinPattern),my_program1)
prog2 = $(patsubst %,$(BinPattern),my_program2)
\end{verbatim}

These variables can then be used throughout the rest of {\tt dir.mk}.  All
platforms support the extensions {\tt .c} for C source files, {\tt .cc} for C++
source files and {\tt .o} for object files.

\subsection{Building C or C++ programs}

For building C or C++ programs use the ``rules'' CExecutable and CXXExecutable
respectively (``rules'' like these are actually make variables which contain a
canned sequence of shell commands).  The general forms are:

\begin{verbatim}
$(prog1): <objects> <lib_depends>
        @(libs="<libs>"; $(CExecutable))

$(prog2): <objects> <lib_depends>
        @(libs="<libs>"; $(CXXExecutable))
\end{verbatim}

Here {\tt prog1} and {\tt prog2} have been assigned using BinPattern as
suggested above, {\tt <objects>} is a list of object files and {\tt <libs>} is
a list of libraries.  {\tt <lib\_depends>} is a list of library dependencies -
for each library provided there should be a make variable both for the library
itself and for putting in the dependencies of a program.  For example the
omnithread library provides a variable {\tt OMNITHREAD\_LIB} which might
contain something like ``{\tt -lomnithread -lpthreads}'' and a variable {\tt
OMNITHREAD\_LIB\_DEPEND} which will contain the full pathname of {\tt
libomnithread.a}.  By specifying such dependencies programs will automatically
be relinked whenever relevant libraries change.

To specify the options that get given to the C or C++ compiler there are
several variables you can use.  The most common one is {\tt DIR\_CPPFLAGS}.
This is used for passing C preprocessor flags (i.e. {\tt -D} and {\tt -I}) to
both the C compiler and the C++ compiler.  The ``{\tt DIR}'' signifies that the
flags are specific to this directory.  These flags will be used in addition to
the normal flags given to the compiler.  If you want to totally override all
the standard CPP flags provided by the import trees you can set the variable
{\tt CPPFLAGS} directly, but this is not recommended unless you know what
you're doing.

Other variables you can set are {\tt CDEBUGFLAGS} or {\tt CXXDEBUGFLAGS} for
setting {\tt -O} and {\tt -g} flags to the compiler, and {\tt COPTIONS} and
{\tt CXXOPTIONS} for setting other options.  Setting these variables will
override any defaults provided by the import trees.  If you want to keep the
defaults but add an extra flag you can use GNU make's ``+='' notation.

As well as the rule to build the executable you should also have a rule to
remove it when ``{\tt omake clean}'' is done and a rule to export it when
``{\tt omake export}'' is done, e.g.:

\begin{verbatim}
clean::
        $(RM) $(prog1) $(prog2)

export:: $(prog1) $(prog2)
        @$(ExportExecutable)
\end{verbatim}

\subsection{Building libraries}

To build a statically-linked (i.e. non-shared) library, use the rule
{\tt StaticLinkLibrary}:

\begin{verbatim}
$(lib): <objects>
        @$(StaticLinkLibrary)
\end{verbatim}

Again you can use {\tt DIR\_CPPFLAGS}, {\tt CDEBUGFLAGS}, etc for controlling
the flags given to the compiler.  To export the library and clean it out you
should also have:

\begin{verbatim}
clean::
        $(RM) $(lib)

export:: $(lib)
        @$(ExportLibrary)
\end{verbatim}


\subsection{Building in subdirectories}

Use the {\tt MakeSubdirs} rule.  Normally you just set the {\tt SUBDIRS}
variable like this:

\begin{verbatim}
SUBDIRS = wib wob

all::
        @$(MakeSubdirs)

export::
        @$(MakeSubdirs)
\end{verbatim}

This will cause both ``{\tt omake all}'' and ``{\tt omake export}'' to go down
into the subdirectories {\tt wib} and {\tt wob}.  If you want to run {\tt
omake} with a different target in the subdirectories you can set the shell
variable {\tt target}.  For example, if you want ``{\tt omake export}'' to
actually perform ``{\tt omake all}'' in each of the subdirectories, do this:

\begin{verbatim}
export::
        @(target=all; $(MakeSubdirs))
\end{verbatim}

In fact you can use different subdirectories for different targets as well by
using the {\tt subdirs} shell variable instead of the {\tt SUBDIRS} make
variable:

\begin{verbatim}
all::
        @(subdirs="a b"; $(MakeSubdirs))

export::
        @(subdirs="c d"; $(MakeSubdirs))
\end{verbatim}

If you want to pass make variable settings down to a subdirectory you can use
the make variable {\tt SUBDIR\_MAKEFLAGS} (as with {\tt SUBDIRS} this can be
overridden with a shell variable, in this case {\tt subdir\_makeflags}).  For
example:

\begin{verbatim}
SUBDIR_MAKEFLAGS = CDEBUGFLAGS=-g MY_MAKE_VARIABLE="foo bar"

all::
        @$(MakeSubdirs)

export::
        @(subdir_makeflags='MY_MAKE_VARIABLE="wib wob"'; $(MakeSubdirs))
\end{verbatim}


\subsection{Local include directories}

One point to note is that when adding local include directories to {\tt
DIR\_CPPFLAGS} you need to be aware of the way the VPATH mechanism works.  For
example, putting:

\begin{verbatim}
DIR_CPPFLAGS = -I../include
\end{verbatim}

will probably not do what you want since only this build tree will be searched
for header files.  If you want to search through all the source trees for the
equivalent include directories you need a ``patsubst'' expression to add the
same relative path to each element of {\tt VPATH}:

\begin{verbatim}
DIR_CPPFLAGS = -I../include $(patsubst %,-I%/../include,$(VPATH))
\end{verbatim}


\section{Creating a New Project Tree}

To create a new project tree all you need is an import/export tree, a source
tree and a set of build trees, one for each platform.  By convention, the
source tree is put in a subdirectory of the import/export tree called {\tt src}
and the build trees are put in subdirectories named {\tt build/}{\it platform}.

Let's assume for your project that you have some idl files and header files as
well as some source files.  Create your top-level directory, say {\tt
/project/wobble/version1.0}.  Then make directories {\tt idl}, {\tt include},
{\tt src} and {\tt build}.  Copy your idl files into {\tt idl} and your header
files into {\tt include}.

For your source files you must decide on an appropriate directory structure
underneath {\tt src} and put the source files there.  You can have any
structure there, but if you want a guide, take a look at the {\tt
/project/omni/version4.0/src} structure.  This has firstly a directory with
tools used in the building process (which are probably unlikely in your project
tree), secondly a directory with ATMos interface files (again unlikely unless
you are writing low-level ATMos code), thirdly a directory with libraries, and
lastly a directory where executable programs are built.

In your source tree you'll need an appropriate {\tt dir.mk} file in each
directory (including one in {\tt src} itself).  At the top level you may not
want ``{\tt omake all}'' to do anything since it is unlikely to do what people
expect (when it gets to building programs it won't use the libraries which have
just been built).  Instead you probably only want an ``{\tt omake export}''
rule at the top level which will export things as they get built.  In this way
executable programs will use the libraries which have just been built and
exported.

Once you have the structure and the {\tt dir.mk} files, you can now create a
build tree for each platform.  Go into the {\tt build} directory and run ``{\tt
obuildtree}''.  Nearly all the the defaults should be correct, but when
prompted for import trees specify your top-level directory ({\tt
/project/wobble/version1.0}), and make sure that the same tree is the export
tree and also that the source tree is the same with {\tt src} appended ({\tt
/project/wobble/version1.0/src}).

Go into the {\it platform} subdirectory and do ``{\tt oshadow}'' then ``{\tt
omake export}''.  If you got your {\tt dir.mk} files correct everything will
now build and be exported to your project import/export tree.

Now that the project import/export tree is set up users can create their own
build trees which import from the project import/export tree, but they should
not export to it from their private build trees.  It may even be worth writing
a project-specific version of {\tt obuildtree} to create a build tree which
automatically imports from both the base OMNI tree and the project tree.

\end{document}
