\documentclass[11pt,twoside,a4paper]{book}
\usepackage[T1]{fontenc}
\usepackage{palatino}

% To make the PostScript version:
%
%  $ latex omniORBpy
%  $ bibtex omniORBpy
%  $ latex omniORBpy
%  $ latex omniORBpy
%  $ latex omniORBpy
%  $ dvips omniORBpy
%
% To make the PDF version (having already made the PS version):
%
%  $ pdflatex omniORBpy
%  $ pdflatex omniORBpy
%
% To make the HTML version, you need HeVeA 1.05.
%
%  $ hevea omniORBpy
%  $ hacha omniORBpy.html
%
% HeVeA and HaChA come from http://pauillac.inria.fr/~maranget/hevea/


\bibliographystyle{alpha}

% Semantic mark-up:
\newcommand{\type}[1]{\texttt{#1}}
\newcommand{\intf}[1]{\texttt{#1}}
\newcommand{\module}[1]{\texttt{#1}}
\newcommand{\code}[1]{\texttt{#1}}
\newcommand{\op}[1]{\texttt{#1()}}
\newcommand{\cmdline}[1]{\texttt{#1}}
\newcommand{\term}[1]{\textit{#1}}

\hyphenation{omni-ORB}
\hyphenation{omni-ORBpy}

% Environment to make important statements stand out:
\newenvironment{statement}%
 {\noindent\begin{minipage}{\textwidth}%
  \vspace{.5\baselineskip}%
  \noindent\rule{\textwidth}{2pt}%
  \vspace{.25\baselineskip}%
  \begin{list}{}{\setlength{\listparindent}{0em}%
                 \setlength{\itemindent}{0em}%
                 \setlength{\leftmargin}{1.5em}%
                 \setlength{\rightmargin}{\leftmargin}%
                 \setlength{\topsep}{0pt}%
                 \setlength{\partopsep}{0pt}}
  \item\relax}
 {\end{list}%
  \vspace{-.25\baselineskip}%
  \noindent\rule{\textwidth}{2pt}%
  \vspace{.5\baselineskip}%
  \end{minipage}}

% Itemize with no itemsep:
\newenvironment{nsitemize}%
 {\begin{itemize}\setlength{\itemsep}{0pt}}%
 {\end{itemize}}


% URL-like things:
\usepackage[T1]{url}
\newcommand{\weburl}{\url}
\newcommand{\email}{\begingroup \urlstyle{rm}\Url}
\newcommand{\file}{\begingroup \urlstyle{tt}\Url}
\newcommand{\envvar}{\begingroup \urlstyle{tt}\Url}
\newcommand{\makevar}{\begingroup \urlstyle{tt}\Url}
\newcommand{\corbauri}{\begingroup \urlstyle{tt}\Url}


% HeVeA barfs at the following:

%BEGIN LATEX

\addtolength{\oddsidemargin}{-0.2in}
\addtolength{\evensidemargin}{-0.6in}
\addtolength{\textwidth}{0.5in}

\newcommand{\dsc}{\discretionary{}{}{}}

\pagestyle{headings}
\setcounter{secnumdepth}{3}
\setcounter{tocdepth}{3}

\usepackage{listings}
\lstdefinelanguage{idl}%
  {keywords={any,attribute,boolean,case,char,const,context,default,double,enum,exception,FALSE,float,in,inout,interface,long,module,Object,octet,oneway,out,raises,readonly,sequence,short,string,struct,switch,TRUE,typedef,unsigned,union,void},%
  sensitive,%
  singlecomment={/*}{*/},%
  commentline=//,%
  stringizer=[b]",%
  directives={define,elif,else,endif,error,if,ifdef,ifndef,line,include,pragma,undef,warning}%
  }[keywords,comments,strings,directives]

\lstset{basicstyle=\ttfamily\small,
        keywordstyle=,
        commentstyle=\itshape,
        labelstyle=\tiny,
        stringspaces=false,
        abovecaptionskip=0pt,
        belowcaptionskip=0pt,
        indent=\parindent,
        fontadjust}

\lstnewenvironment{idllisting}{\lstset{language=idl}}{}
\lstnewenvironment{cxxlisting}{\lstset{language=C++}}{}
\lstnewenvironment{makelisting}{\lstset{language=[gnu]make}}{}
\lstnewenvironment{pylisting}{\lstset{language=python}}{}

%END LATEX

% These things make up for HeVeA's lack of understanding:

%HEVEA\newcommand{\dsc}{}
%HEVEA\newcommand{\vfill}{}
%HEVEA\newcommand{\mainmatter}{}
%HEVEA\newcommand{\backmatter}{}
%HEVEA\newcommand{\lstset}[1]{}
%HEVEA\newcommand{\textbackslash}{$\backslash$}
%HEVEA\usepackage{verbatim}
%HEVEA\newenvironment{idllisting}{\verbatim}{\endverbatim}
%HEVEA\newenvironment{cxxlisting}{\verbatim}{\endverbatim}
%HEVEA\newenvironment{makelisting}{\verbatim}{\endverbatim}
%HEVEA\newenvironment{pylisting}{\verbatim}{\endverbatim}


% Hyperref things for pdf and html:
\usepackage{hyperref}

\newif\ifpdf
\ifx\pdfoutput\undefined
  \pdffalse
\else
  \pdfoutput=1
  \pdftrue
\fi
\ifpdf
  \hypersetup{colorlinks,citecolor=red,urlcolor=blue}
\fi


% Finally, the set-up is almost over, and we can start the document:

\begin{document}

\pagenumbering{roman}
\pagestyle{empty}


\begin{center}

\vfill

{ \Huge
The omniORBpy version 1.0\\[4mm]
User's Guide
}

\vfill

{ \Large
Duncan Grisby\\
{\normalsize (\textit{email: \href{mailto:dgrisby@uk.research.att.com}%
                                  {\email{dgrisby@uk.research.att.com}}})}%
                                  \\[2ex]
%
AT\&T Laboratories Cambridge\\
}


\vfill
\vfill
June 2000
\vfill

\end{center}

\clearpage

{\Large \bf Changes and Additions, June 2000}
\begin{itemize}
\item Brand new manual
\end{itemize}


\cleardoublepage

%HEVEA{\Large \bf Contents}
\tableofcontents

\cleardoublepage

\pagestyle{headings}

\pagenumbering{arabic}

\mainmatter

%%%%%%%%%%%%%%%%%%%%%%%%%%%%%%%%%%%%%%%%%%%%%%%%%%%%%%%%%%%%%%%%%%%%%%
\chapter{Introduction}
%%%%%%%%%%%%%%%%%%%%%%%%%%%%%%%%%%%%%%%%%%%%%%%%%%%%%%%%%%%%%%%%%%%%%%

omniORBpy is an Object Request Broker (ORB) that implements the CORBA
2.3 Python mapping~\cite{pythonmapping}. It is designed for use with
omniORB 3, but it can also be used with omniORB 2.8. If you use
omniORB 3.0, the full POA functionality is available; with omniORB 2.8
many POA functions are not supported.

This user guide tells you how to use omniORBpy to develop CORBA
applications using Python. It assumes a basic understanding of CORBA,
and of the Python mapping. Unlike most CORBA standards, the Python
mapping document is small, and quite easy to follow.

This manual contains all you need to know about omniORB in order to
use omniORBpy. Some sections are repeated from the omniORB manual.

In this chapter, we give an overview of the main features of omniORBpy
and what you need to do to setup your environment to run it.

\section{Features}

\subsection{CORBA 2.3 compliant}

omniORB implements the Internet Inter-ORB Protocol (IIOP).  This
protocol provides omniORB the means of achieving interoperability with
the ORBs implemented by other vendors. In fact, this is the native
protocol used by omniORB for the communication amongst its objects
residing in different address spaces.

Moreover, the IDL to Python language mapping provided by omniORBpy
conforms to the 2.3 Python mapping specification\footnote{As detailed
in section~\ref{sec:missing}, a number of features are missing.}.


\subsection{Multithreading}

omniORBpy is fully multithreaded\footnote{This means that your Python
interpreter must have been built with thread support.}. To achieve low
IIOP call overhead, unnecessary call-multiplexing is eliminated. At
any time, there is at most one call in-flight in each communication
channel between two address spaces. To do so without limiting the
level of concurrency, new channels connecting the two address spaces
are created on demand and cached when there are more concurrent calls
in progress. Each channel is served by a dedicated thread. This
arrangement provides maximal concurrency and eliminates any thread
switching in either of the address spaces to process a
call. Furthermore, to maximise the throughput in processing large call
arguments, large data elements are sent as soon as they are processed
while the other arguments are being marshalled.

\subsection{Missing features}
\label{sec:missing}

omniORBpy is not a complete implementation of the CORBA 2.3 core.  The
following is a list of the missing features.

\begin{itemize}

\item omniORB does not have its own Interface Repository. However, it
can act as a client to an IR\footnote{See section~\ref{sec:ifrstubs}
for a note about using an interface repository.}.

\item The IDL types wchar, wstring, fixed, and valuetype are not
supported in this release.

\item The \type{PortableServer.Current} interface is not yet
supported.

\item DII and DSI are not supported. However, since both Python code
and IDL can be generated and used at run-time, this is not a
significant restriction.

\end{itemize}

These features may be implemented in the short to medium term. It is
best to check out the latest status on the omniORB home page
(\weburl{http://www.uk.research.att.com/omniORB/}).


\section{Setting up your environment}
\label{sec:setup}

omniORBpy relies on the omniORB C++ libraries. If you are building
from source, you must first build omniORB itself, as detailed in the
omniORB documentation. After that, you can build the omniORBpy
distribution, according to the instructions in the release notes.


\subsection{Paths}

For Python to find omniORBpy, you must add two directories to the
\envvar{PYTHONPATH} environment variable. The \file{lib/python}
directory contains platform-independent Python code; the
\file{lib/}\texttt{\$}\file{FARCH} directory contains
platform-specific binaries, where \envvar{FARCH} is the name of your
platform, such as \file{i586_linux_glibc} or \file{x86_win32}.

On Unix platforms, set \envvar{PYTHONPATH} with a command like:

\begin{verbatim}
   export PYTHONPATH=$PYTHONPATH:$TOP/lib/python:$TOP/lib/$FARCH
\end{verbatim}

On Windows, use

\begin{verbatim}
   set PYTHONPATH=%PYTHONPATH%;%TOP%\lib\python;%TOP%\lib\x86_win32
\end{verbatim}

(Where the \envvar{TOP} environment variable is the root of your
omniORB tree.)

You should also add the \file{bin/}\texttt{\$}\file{FARCH} directory
to your \envvar{PATH}, so you can run the IDL compiler, omniidl.


\subsection{Configuration file}

\begin{itemize}

\item On Unix platforms, the omniORB runtime looks for the environment
variable \envvar{OMNIORB_CONFIG}. If this variable is defined, it
contains the pathname of the omniORB configuration file. If the
variable is not set, omniORB will use the compiled-in pathname to
locate the file.

\item On Win32 platforms (Windows NT, 2000, 95, 98), omniORB first
checks the environment variable \envvar{OMNIORB_CONFIG} to obtain the
pathname of the configuration file. If this is not set, it then
attempts to obtain configuration data in the system registry. It
searches for the data under the key
%BEGIN LATEX
\file{HKEY_LOCAL_MACHINE\SOFTWARE\ORL\omniORB\2.0}.
%END LATEX
%HEVEA \verb|HKEY_LOCAL_MACHINE\SOFTWARE\ORL\omniORB\2.0|.

\end{itemize}

The configuration file is used to obtain an object reference for the
Naming Service, via either a stringified IOR or an Interoperable
Naming Service URI. When using omniORB 3, the entry in the
configuration file should be specified in the form:

\begin{verbatim}
   ORBInitRef NameService=<URI for the Naming Service>
\end{verbatim}

The easiest way of specifying the Naming Service is with a
\corbauri{corbaname:} URI, as described in
section~\ref{sec:corbaname}.

Comments in the configuration file should be prefixed with a
`\texttt{\#}' character.

On Win32 platforms, the naming service reference can be placed in the
system registry, in the (string) value \texttt{ORBInitRef}, under the
key
%BEGIN LATEX
\file{HKEY_LOCAL_MACHINE\SOFTWARE\ORL\omniORB\2.0}.
%END LATEX
%HEVEA \verb|HKEY_LOCAL_MACHINE\SOFTWARE\ORL\omniORB\2.0|.


\subsubsection{omniORB 2.8 configuration file entries}

\texttt{ORBInitRef} is new with omniORB 3. Under omniORB 2.8, you
must use the older format (which is still supported by omniORB 3):

\begin{verbatim}
   NAMESERVICE <IOR for the Naming Service>
\end{verbatim}

Two other entries are also supported, but with omniORB 3 are made
obsolete by the Interoperable Naming Service URIs:

\begin{verbatim}
   ORBInitialHost <hostname string>
   ORBInitialPort <port number (1-65535)>
\end{verbatim}

The corresponding entries under the Win32 system registry are the keys
named \texttt{ORBInitialHost} and \texttt{ORBInitialPort}.

The two entries provide information to the ORB to locate a
(proprietary) bootstrap service at runtime. The bootstrap service is
able to return the initial object reference for the Naming Service and
others.



%%%%%%%%%%%%%%%%%%%%%%%%%%%%%%%%%%%%%%%%%%%%%%%%%%%%%%%%%%%%%%%%%%%%%%
\chapter{The Basics}
%%%%%%%%%%%%%%%%%%%%%%%%%%%%%%%%%%%%%%%%%%%%%%%%%%%%%%%%%%%%%%%%%%%%%%
\label{chap:basics}

In this chapter, we go through three examples to illustrate the
practical steps to use omniORBpy. By going through the source code of
each example, the essential concepts and APIs are introduced. If you
have no previous experience with using CORBA, you should study this
chapter in detail. There are pointers to other essential documents you
should be familiar with.

If you have experience with using other ORBs, you should still go
through this chapter because it provides important information about
the features and APIs that are necessarily omniORB specific.


\section{The Echo example}

We use an example which is similar to the one used in the omniORB
manual. We define an interface, called \intf{Example::Echo}, as
follows:

\begin{idllisting}
// echo_example.idl
module Example {
  interface Echo {
    string echoString(in string mesg);
  };
};
\end{idllisting}

The important difference from the omniORB Echo example is that our
\intf{Echo} interface is declared within an IDL module named
\module{Example}. The reason for this will become clear in a moment.

If you are new to IDL, you can learn about its syntax in Chapter 3 of
the CORBA specification 2.3~\cite{corba23-spec}. For the moment, you
only need to know that the interface consists of a single operation,
\op{echoString}, which takes a string as an argument and returns a
copy of the same string.

The interface is written in a file, called \file{echo_example.idl}. It
is part of the CORBA standard that all IDL files should have the
extension `\file{.idl}', although omniORB does not enforce this.


\section{Generating the Python stubs}
\label{sec:generatingStubs}

From the IDL file, we use the IDL compiler, omniidl, to produce the
Python stubs for that IDL. The stubs contain Python declarations for
all the interfaces and types declared in the IDL, as required by the
Python mapping. It is possible to generate stubs dynamically at
run-time, as described in section~\ref{sec:importIDL}, but it is more
efficient to generate them statically.

To generate the stubs, we use a command line like

\begin{quote}
\cmdline{omniidl -bpython echo\_example.idl}
\end{quote}

\noindent As required by the standard, that produces two Python
packages derived from the module name \module{Example}. Directory
\file{Example} contains the client-side definitions (and also the type
declarations if there were any); directory \file{Example__POA}
contains the server-side skeletons. This explains the difficulty with
declarations at IDL global scope; section~\ref{sec:globalIDL} explains
how to access global declarations.

If you look at the Python code in the two packages, you will see that
they are almost empty. They simply import the
\file{echo_example_idl.py} file, which is where both the client and
server side declarations actually live. This arrangement is so that
omniidl can easily extend the packages if other IDL files add
declarations to the same IDL modules.


\section{Object References and Servants}

We contact a CORBA object through an \term{object reference}. The
actual implementation of a CORBA object is termed a \term{servant}.

Object references and servants are quite separate entities, and it is
important not to confuse the two. Client code deals purely with object
references, so there can be no confusion; object implementation code
must deal with both object references and servants. You will get a
run-time error if you use a servant where an object reference is
expected, or vice-versa.


\section{Example 1 --- Colocated client and servant}
\label{sec:eg1}

In the first example, both the client and servant are in the same
address space. The next sections show how the client and servant can
be split between different address spaces.

First, the code:

\lstset{labelstep=1,gobble=4}
\begin{pylisting}
 1  #!/usr/bin/env python
 2  
 3  import sys
 4  from omniORB import CORBA, PortableServer
 5  import Example, Example__POA
 6  
 7  class Echo_i (Example__POA.Echo):
 8      def echoString(self, mesg):
 9          print "echoString() called with message:", mesg
10          return mesg
11  
12  orb = CORBA.ORB_init(sys.argv, CORBA.ORB_ID)
13  poa = orb.resolve_initial_references("RootPOA")
14  
15  ei = Echo_i()
16  eo = ei._this()
17  
18  poaManager = poa._get_the_POAManager()
19  poaManager.activate()
20  
21  message = "Hello"
22  result  = eo.echoString(message)
23
24  print "I said '%s'. The object said '%s'." % (message,result)
\end{pylisting}
\lstset{labelstep=0,gobble=0}

The example illustrates several important interactions among the ORB,
the POA, the servant, and the client. Here are the details:

\subsection{Imports}

\begin{description}

\item[Line 3]\mbox{}\\
%
Import the \module{sys} module to access \code{sys.argv}.

\item[Line 4]\mbox{}\\
%
Import omniORB's implementations of the \module{CORBA} and
\module{PortableServer} modules. The standard requires that these
modules are available outside of any package, so you can also do

\begin{pylisting}
import CORBA, PortableServer
\end{pylisting}

\noindent Explicitly specifying omniORB is useful if you have more
than one Python ORB installed.

\item[Line 5]\mbox{}\\
%
Import the client-side stubs and server-side skeletons generated for
IDL module \module{Example}.

\end{description}


\subsection{Servant class definition}

\begin{description}

\item[Lines 7--10]\mbox{}\\
%
For interface \intf{Example::Echo}, omniidl produces a skeleton class
named \type{Example\_\_POA.Echo}. Here we define an implementation
class, \type{Echo\_i}, which derives from the skeleton class.

There is little constraint on how you design your implementation
class, except that it has to inherit from the skeleton class and must
implement all of the operations declared in the IDL. Note that since
Python is a dynamic language, errors due to missing operations and
operations with incorrect type signatures are only reported when
someone tries to call those operations.

\end{description}


\subsection{ORB initialisation}

\begin{description}

\item[Line 12]\mbox{}\\
%
The ORB is initialised by calling the \op{CORBA.ORB\_init} function.
\op{ORB\_\dsc{}init} is passed a list of command-line arguments, and
an ORB identifier. The ORB identifier should be `omniORB3' or
`omniORB2', depending on which version of omniORB you are using. It is
usually best to use \code{CORBA.ORB\_ID}, which is initialised to a
suitable string.

\op{ORB\_init} processes any command-line arguments which begin with
the string `\cmdline{-ORB}', and removes them from the argument
list. See section~\ref{sec:ORBargs} for details. If any arguments are
invalid, or other initialisation errors occur (such as errors in the
configuration file), the \code{CORBA.INITIALIZE} exception is raised.

\end{description}

\subsection{Obtaining the Root POA}

\begin{description}

\item[Line 13]\mbox{}\\
%
To activate our servant object and make it available to clients, we
must register it with a POA. In this example, we use the \term{Root
POA}, rather than creating any child POAs. The Root POA is found with
\op{orb.resolve\_initial\_\dsc{}references}.

A POA's behaviour is governed by its \term{policies}. The Root POA has
suitable policies for many simple servers. Chapter 11 of the CORBA 2.3
specification \cite{corba23-spec} has details of all the POA policies
which are available.

When omniORBpy is used with omniORB 2.8, \emph{only} the Root POA is
available (and is mapped to the omniORB BOA). You cannot create child
POAs or alter policies.

\end{description}


\subsection{Object initialisation}

\begin{description}

\item[Line 15]\mbox{}\\
%
An instance of the Echo servant object is created.

\item[Line 16]\mbox{}\\
%
The object is implicitly activated in the Root POA, and an object
reference is returned, using the \op{\_this} method.

One of the important characteristics of an object reference is that it
is completely location transparent. A client can invoke on the object
using its object reference without any need to know whether the
servant object is colocated in the same address space or is in a
different address space.

In the case of colocated client and servant, omniORB is able to
short-circuit the client calls so they do not involve IIOP. The calls
still go through the POA, however, so the various POA policies affect
local calls in the same way as remote ones. This optimisation is
applicable not only to object references returned by \op{\_this}, but
to any object references that are passed around within the same
address space or received from other address spaces via IIOP calls.

\end{description}


\subsection{Activating the POA}

\begin{description}

\item[Lines 18--19]\mbox{}\\
%
POAs are initially in the \term{holding} state, meaning that incoming
requests are blocked. Lines 18 and 19 acquire a reference to the POA's
POA manager, and use it to put the POA into the \term{active} state.
Incoming requests are now served.

\end{description}


\subsection{Performing a call}

\begin{description}

\item[Line 22]\mbox{}\\
%
At long last, we can call the object's \op{echoString} operation.
Even though the object is local, the operation goes through the ORB
and POA, so the types of the arguments can be checked, and any mutable
arguments can be copied. This ensures that the semantics of local and
remote calls are identical. If any of the arguments (or return values)
are of the wrong type, a \code{CORBA.BAD\_PARAM} exception is raised.

\end{description}


\section{Example 2 --- Different Address Spaces}

In this example, the client and the object implementation reside in
two different address spaces. The code of this example is almost the
same as the previous example. The only difference is the extra work
which needs to be done to pass the object reference from the object
implementation to the client.

The simplest (and quite primitive) way to pass an object reference
between two address spaces is to produce a \term{stringified} version
of the object reference and to pass this string to the client as a
command-line argument.  The string is then converted by the client
into a proper object reference.  This method is used in this
example. In the next example, we shall introduce a better way of
passing the object reference using the CORBA Naming Service.


\subsection{Object Implementation: Generating a Stringified Object Reference}

\lstset{labelstep=1,gobble=4}
\begin{pylisting}
 1  #!/usr/bin/env python
 2  
 3  import sys
 4  from omniORB import CORBA, PortableServer
 5  import Example, Example__POA
 6  
 7  class Echo_i (Example__POA.Echo):
 8      def echoString(self, mesg):
 9          print "echoString() called with message:", mesg
10          return mesg
11  
12  orb = CORBA.ORB_init(sys.argv, CORBA.ORB_ID)
13  poa = orb.resolve_initial_references("RootPOA")
14  
15  ei = Echo_i()
16  eo = ei._this()
17  
18  print orb.object_to_string(eo)
19  
20  poaManager = poa._get_the_POAManager()
21  poaManager.activate()
22  
23  orb.run()
\end{pylisting}
\lstset{labelstep=0,gobble=0}

Up until line 18, this example is identical to the colocated case. On
line 18, the ORB's \op{object\_to\_string} operation is called. This
results in a string starting with the signature `IOR:' and followed by
some hexadecimal digits. All CORBA 2 compliant ORBs are able to
convert the string into its internal representation of a so-called
Interoperable Object Reference (IOR). The IOR contains the location
information and a key to uniquely identify the object implementation
in its own address space\footnote{Notice that the object key is not
globally unique across address spaces.}. From the IOR, an object
reference can be constructed.

After the POA has been activated, \op{orb.run} is called. Since
omniORB is fully multi-threaded, it is not actually necessary to call
\op{orb.run} for operation dispatch to happen---if the main program
had some other work to do, it could do so, and remote invocations
would be dispatched in separate threads. However, in the absence of
anything else to do, \op{orb.run} is called so the thread blocks
rather than exiting immediately when the end-of-file is reached.
\op{orb.run} stays blocked until the ORB is shut down.

\vspace{\baselineskip}% What's going wrong here?

\subsection{Client: Using a Stringified Object Reference}
\label{clnt2}

\lstset{labelstep=1,gobble=4}
\begin{pylisting}
 1  #!/usr/bin/env python
 2  
 3  import sys
 4  from omniORB import CORBA
 5  import Example
 6  
 7  orb = CORBA.ORB_init(sys.argv, CORBA.ORB_ID)
 8  
 9  ior = sys.argv[1]
10  obj = orb.string_to_object(ior)
11  
12  eo = obj._narrow(Example.Echo)
13  
14  if eo is None:
15      print "Object reference is not an Example::Echo"
16      sys.exit(1)
17  
18  message = "Hello from Python"
19  result  = eo.echoString(message)
20  
21  print "I said '%s'. The object said '%s'." % (message,result)
\end{pylisting}
\lstset{labelstep=0,gobble=0}

The stringified object reference is passed to the client as a
command-line argument\footnote{The code does not check that there is
actually an IOR on the command line!}. The client uses the ORB's
\op{string\_to\_object} function to convert the string into a generic
object reference (\type{CORBA.Object}).

On line 12, the object's \op{\_narrow} function is called to convert
the \type{CORBA.\dsc{}Object} reference into an \type{Example.Echo}
reference. If the IOR was not actually of type \type{Example.Echo}, or
something derived from it, \op{\_narrow} returns \code{None}.

In fact, since Python is a dynamically-typed language,
\op{string\_to\_object} is often able to return an object reference of
a more derived type than \type{CORBA.\dsc{}Object}. See
section~\ref{sec:narrowing} for details.

\vspace{\baselineskip}% What's up here?


\subsection{System exceptions}

The keep it short, the client code shown above performs no exception
handling. A robust client (and server) should do, since there are a
number of system exceptions which can arise.

As already mentioned, \op{ORB\_init} can raise the
\code{CORBA.INITIALIZE} exception if the command line arguments or
configuration file are invalid.  \op{string\_to\_\dsc{}object} can
raise two exceptions: if the string is not an IOR (or a valid URI with
omniORB 3), it raises \code{CORBA.BAD\_PARAM}; if the string looks
like an IOR, but contains invalid data, is raises
\code{CORBA.MARSHAL}.

The call to \op{echoString} can result in any of the CORBA system
exceptions, since any exceptions not caught on the server side are
propagated back to the client. Even if the implementation of
\op{echoString} does not raise any system exceptions itself, failures
in invoking the operation can cause a number of exceptions. First, if
the server process cannot be contacted, a \code{CORBA.COMM\_FAILURE}
exception is raised. Second, if the server process \emph{can} be
contacted, but the object in question does not exist there, a
\code{CORBA.OBJECT\_NOT\_EXIST} exception is raised. Various events
can also cause \code{CORBA.TRANSIENT} to be raised. If that occurs,
omniORB's default behaviour is to automatically retry the invocation,
with exponential back-off.

As explained later in section~\ref{sec:narrowing}, the call to
\op{\_narrow} may also involve a call to the object to confirm its
type. This means that \op{\_narrow} can also raise
\code{CORBA.COMM\_FAILURE}, \code{CORBA.OBJECT\_NOT\_EXIST}, and
\code{CORBA.TRANSIENT}.

Section~\ref{sec:exHandlers} describes how exception handlers can be
installed for all the various system exceptions, to avoid surrounding
all code with \code{try}\dots\code{except} blocks.


\subsection{Lifetime of a CORBA object}

CORBA objects are either \term{transient} or \term{persistent}. The
majority are transient, meaning that the lifetime of the CORBA object
(as contacted through an object reference) is the same as the lifetime
of its servant object. Persistent objects can live beyond the
destruction of their servant object, the POA they were created in, and
even their process. Persistent objects are, of course, only
contactable when their associated servants are active, or can be
activated by their POA with a servant manager\footnote{The POA itself
can be activated on demand with an adapter activator.}. A reference to
a persistent object can be published, and will remain valid even if
the server process is restarted.

A POA's Lifespan Policy determines whether objects created within it
are transient or persistent. The Root POA has the \code{TRANSIENT}
policy. (Note that since only the Root POA is available when using
omniORBpy with omniORB 2.8, it is not possible to create persistent
objects in that environment.)

An alternative to creating persistent objects is to register object
references in a \term{naming service} and bind them to fixed
pathnames. Clients can bind to the object implementations at runtime
by asking the naming service to resolve the pathnames to the object
references. CORBA defines a standard naming service, which is a
component of the Common Object Services (COS)~\cite{corbaservices},
that can be used for this purpose. The next section describes an
example of how to use the COS Naming Service.



\section{Example 3 --- Using the Naming Service}
\label{sec:usingNS}

In this example, the object implementation uses the Naming
Service~\cite{corbaservices} to pass on the object reference to the
client.  This method is far more practical than using stringified
object references. The full listings of the server and client are
below.

The names used by the Naming service consist of a sequence of
\term{name components}. Each name component has an \term{id} and a
\term{kind} field, both of which are strings. All name components
except the last one are bound to \term{naming contexts}. A naming
context is analogous to a directory in a filing system: it can contain
names of object references or other naming contexts. The last name
component is bound to an object reference.

Sequences of name components can be represented as a flat string,
using `.' to separate the id and kind fields, and `/' to separate name
components from each other\footnote{There are escaping rules to cope
with id and kind fields which contain `.' and `/' characters. See
chapter~\ref{chap:ins} of this manual, and chapter 3 of the CORBA
services specification, as updated for the Interoperable Naming
Service~\cite{inschapters}.}. In our example, the Echo object
reference is bound to the stringified name
`\file{test.my_context/ExampleEcho.Object}'.

The kind field is intended to describe the name in a
syntax-independent way. The naming service does not interpret, assign,
or manage these values. However, both the name and the kind attribute
must match for a name lookup to succeed. In this example, the kind
values for \file{test} and \file{ExampleEcho} are chosen to be
`\file{my_context}' and `\file{Object}' respectively. This is an
arbitrary choice as there is no standardised set of kind values.


\subsection{Obtaining the Root Context object reference}
\label{resolveinit}

The initial contact with the Naming Service can be established via the
\term{root} context. The object reference to the root context is
provided by the ORB and can be obtained by calling
\op{resolve\_initial\_references}. The following code fragment shows
how it is used:

\begin{pylisting}
import CosNaming
orb = CORBA.ORB_init(sys.argv, CORBA.ORB_ID)
obj = orb.resolve_initial_references("NameService");
cxt = obj._narrow(CosNaming.NamingContext)
\end{pylisting}

Remember, omniORB constructs its internal list of initial references
at initialisation time using the information provided in the
configuration file \file{omniORB.cfg}, or given on the command
line. If this file is not present, the internal list will be empty and
\op{resolve\_initial\_references} will raise a
\code{CORBA.ORB.\dsc{}InvalidName} exception.

Note that, like \op{string\_to\_object},
\op{resolve\_initial\_references} returns base \type{CORBA.Object}, so
we should narrow it to the interface we want. In this case, we want
\type{CosNaming.NamingContext}\footnote{If you are on-the-ball, you
will have noticed that we didn't call \op{\_narrow} when resolving the
Root POA. The reason it is safe to miss it out is given in
section~\ref{sec:narrowing}.}.


\subsection{The Naming Service interface}

It is beyond the scope of this chapter to describe in detail the
Naming Service interface. You should consult the CORBA services
specification~\cite{corbaservices} (chapter 3).

\subsection{Server code}

Hopefully, the server code is self-explanatory:

\begin{pylisting}
#!/usr/bin/env python
import sys
from omniORB import CORBA, PortableServer
import CosNaming, Example, Example__POA

# Define an implementation of the Echo interface
class Echo_i (Example__POA.Echo):
    def echoString(self, mesg):
        print "echoString() called with message:", mesg
        return mesg

# Initialise the ORB and find the root POA
orb = CORBA.ORB_init(sys.argv, CORBA.ORB_ID)
poa = orb.resolve_initial_references("RootPOA")

# Create an instance of Echo_i and an Echo object reference
ei = Echo_i()
eo = ei._this()

# Obtain a reference to the root naming context
obj         = orb.resolve_initial_references("NameService")
rootContext = obj._narrow(CosNaming.NamingContext)

if rootContext is None:
    print "Failed to narrow the root naming context"
    sys.exit(1)

# Bind a context named "test.my_context" to the root context
name = [CosNaming.NameComponent("test", "my_context")]
try:
    testContext = rootContext.bind_new_context(name)
    print "New test context bound"
    
except CosNaming.NamingContext.AlreadyBound, ex:
    print "Test context already exists"
    obj = rootContext.resolve(name)
    testContext = obj._narrow(CosNaming.NamingContext)
    if testContext is None:
        print "test.mycontext exists but is not a NamingContext"
        sys.exit(1)

# Bind the Echo object to the test context
name = [CosNaming.NameComponent("ExampleEcho", "Object")]
try:
    testContext.bind(name, eo)
    print "New ExampleEcho object bound"

except CosNaming.NamingContext.AlreadyBound:
    testContext.rebind(name, eo)
    print "ExampleEcho binding already existed -- rebound"

# Activate the POA
poaManager = poa._get_the_POAManager()
poaManager.activate()

# Block for ever (or until the ORB is shut down)
orb.run()
\end{pylisting}


\subsection{Client code}

Hopefully the client code is self-explanatory too:

\begin{pylisting}
#!/usr/bin/env python
import sys
from omniORB import CORBA
import CosNaming, Example

# Initialise the ORB
orb = CORBA.ORB_init(sys.argv, CORBA.ORB_ID)

# Obtain a reference to the root naming context
obj         = orb.resolve_initial_references("NameService")
rootContext = obj._narrow(CosNaming.NamingContext)

if rootContext is None:
    print "Failed to narrow the root naming context"
    sys.exit(1)

# Resolve the name "test.my_context/ExampleEcho.Object"
name = [CosNaming.NameComponent("test", "my_context"),
        CosNaming.NameComponent("ExampleEcho", "Object")]
try:
    obj = rootContext.resolve(name)

except CosNaming.NamingContext.NotFound, ex:
    print "Name not found"
    sys.exit(1)

# Narrow the object to an Example::Echo
eo = obj._narrow(Example.Echo)

if (eo is None):
    print "Object reference is not an Example::Echo"
    sys.exit(1)

# Invoke the echoString operation
message = "Hello from Python"
result  = eo.echoString(message)

print "I said '%s'. The object said '%s'." % (message,result)
\end{pylisting}




\section{Global IDL definitions}
\label{sec:globalIDL}

As we have seen, the Python mapping maps IDL modules to Python
packages with the same name. This poses a problem for IDL declarations
at global scope. Global declarations are generally a bad idea since
they make name clashes more likely, but they must be supported.

Since Python does not have a concept of a global scope (only a
per-module global scope, which is dangerous to modify), global
declarations are mapped to a specially named Python package. By
default, this package is named \module{\_GlobalIDL}, with skeletons in
\module{\_GlobalIDL\_\_POA}. The package name may be changed with
omniidl's \cmdline{-Wbglobal} option, described in
section~\ref{sec:Wbglobal}. The omniORB C++ Echo example, with IDL:

\begin{idllisting}
interface Echo {
  string echoString(in string mesg);
};
\end{idllisting}

\noindent can therefore be supported with code like

\begin{pylisting}
#!/usr/bin/env python

import sys
from omniORB import CORBA
import _GlobalIDL

orb = CORBA.ORB_init(sys.argv, CORBA.ORB_ID)

ior = sys.argv[1]
obj = orb.string_to_object(ior)
eo  = obj._narrow(_GlobalIDL.Echo)

message = "Hello from Python"
result  = eo.echoString(message)
print "I said '%s'. The object said '%s'" % (message,result)
\end{pylisting}



%%%%%%%%%%%%%%%%%%%%%%%%%%%%%%%%%%%%%%%%%%%%%%%%%%%%%%%%%%%%%%%%%%%%%%
\chapter{Python language mapping issues}
%%%%%%%%%%%%%%%%%%%%%%%%%%%%%%%%%%%%%%%%%%%%%%%%%%%%%%%%%%%%%%%%%%%%%%

omniORBpy adheres to the standard Python mapping~\cite{pythonmapping},
so there is no need to describe the mapping here. This chapter
outlines a number of issues which are not addressed by the standard
(or are optional), and how they are resolved in omniORBpy.

\section{Narrowing object references}
\label{sec:narrowing}

As explained in chapter~\ref{chap:basics}, whenever you receive an
object reference declared to be base \type{CORBA::Object}, such as
from \op{NamingContext::resolve} or
\op{ORB::\dsc{}string\_to\_object}, you should narrow the reference to
the type you require. You might think that since Python is a
dynamically typed language, narrowing should never be necessary.
Unfortunately, although omniORBpy often generates object references
with the right types, it cannot do so in all circumstances.

The rules which govern when narrowing is required are quite complex.
To be totally safe, you can \emph{always} narrow object references to
the type you are expecting. The advantages of this approach are that
it is simple and that it is guaranteed to work with all Python ORBs.

The disadvantage with calling narrow for all received object
references is that much of the time it is guaranteed not to be
necessary. If you understand the situations in which narrowing
\emph{is} necessary, you can avoid spurious narrowing.


\subsection{The gory details}

When object references are transmitted (or stored in stringified
IORs), they contain a single type identifier string, termed the
\term{repository id}. Normally, the repository id represents the most
derived interface of the object. However, it is also permitted to be
the empty string, or to refer to an interface higher up the
inheritance hierarchy. To give a concrete example, suppose there are
two IDL files:

\begin{idllisting}
// a.idl
module M1 {
  interface A {
    void opA();
  };
};
\end{idllisting}

\begin{idllisting}
// b.idl
#include "a.idl"
module M2 {
  interface B : M1::A {
    void opB();
  };
};
\end{idllisting}


\noindent A reference to an object with interface \intf{B} will
normally contain the repository id `\texttt{IDL:M2/B:1.0}'\footnote{It
is possible to change the repository id strings associated with
particular interfaces using the \code{ID}, \code{version} and
\code{prefix} pragmas.}. It is also permitted to have an empty
repository id, or the id `\texttt{IDL:M1/A:1.0}'.
`\texttt{IDL:M1/A:1.0}' is unlikely unless the server is being
deliberately obtuse.

Whenever omniORBpy receives an object reference from
somewhere---either as a return value or as an operation argument---it
has a particular \term{target} interface in mind, which it compares
with the repository id it has received. A target of base
\intf{CORBA::Object} is just one (common) case. For example, in the
following IDL:

\begin{idllisting}
// c.idl
#include "a.idl"
module M3 {
  interface C {
    Object getObj();
    M1::A  getA();
  };
};
\end{idllisting}

\noindent the target interface for \code{getObj}'s return value is
\intf{CORBA::Object}; the target interface for \code{getA}'s return
value is \intf{M1::A}.

omniORBpy uses the result of comparing the received and target
repository ids to determine the type of the object reference it
creates. The object reference has either the type of the received
reference, or the target type, according to this table:

\begin{center}
\begin{tabular}{lp{.65\textwidth}|l}

\multicolumn{2}{l|}{\textbf{Case}} &
                               \makebox[0pt][l]{\textbf{Objref Type}}\\\hline

1. & The received id is the same as the target id            & received\\\hline

2. & The received id is not the same as the target id, but the
     ORB knows that the received interface is derived from
     the target interface                                    & received\\\hline

3. & The received id is unknown to the ORB                   & target\\\hline

4. & The received id is not the same as the target id, and the
     ORB knows that the received interface is \emph{not}
     derived from the target interface                       & target

\end{tabular}
\end{center}

Cases 1 and 2 are the most common. Case 2 explains why it is not
necessary to narrow the result of calling
\code{resolve\_initial\_references("RootPOA")}: the return is always
of the known type \type{PortableServer.POA}, which is derived from the
target type of \type{CORBA.Object}.

Case 3 is also quite common. Suppose a client knows about IDL modules
\module{M1} and \module{M3} from above, but not module \module{M2}.
When it calls \op{getA} on an instance of \type{M3::C}, the return
value may validly be of type \type{M2::B}, which it does not know. By
creating an object reference of type \type{M1::A} in this case, the
client is still able to call the object's \op{opA} operation. On the
other hand, if \op{getObj} returns an object of type \type{M2::B}, the
ORB will create a reference to base \type{CORBA::Object}, since that
is the target type.

Note that the ORB \emph{never} rejects an object reference due to it
having the wrong type. Even if it knows that the received id is not
derived from the target interface (case 4), it might be the case that
the object actually has a more derived interface, which is derived
from both the type it is claiming to be \emph{and} the target type.
That is, of course, extremely unlikely.

In cases 3 and 4, the ORB confirms the type of the object by calling
\op{\_is\_a} just before the first invocation on the object. If it
turns out that the object is not of the right type after all, the
\code{CORBA.INV\_OBJREF} exception is raised. The alternative to this
approach would be to check the types of object references when they
were received, rather than waiting until the first invocation. That
would be inefficient, however, since it is quite possible that a
received object reference will never be used. It may also cause
objects to be activated earlier than expected.

In summary, whenever your code receives an object reference, you
should bear in mind what omniORBpy's idea of the target type is. You
must not assume that the ORB will always correctly figure out a more
derived type than the target. One consequence of this is that you must
always narrow a plain \type{CORBA::Object} to a more specific type
before invoking on it\footnote{Unless you are invoking pseudo
operations like \op{\_is\_a} and \op{\_non\_existent}.}. You
\emph{can} assume that the object reference you receive is of the
target type, or something derived from it, although the object it
refers to may turn out to be invalid. The fact that omniORBpy often
\emph{is} able figure out a more derived type than the target is only
useful when using the Python interactive command line.




\section{Support for Any values}

In statically typed languages, such as C++, Anys can only be used with
built-in types and IDL-declared types for which stubs have been
generated. If, for example, a C++ program receives an Any containing a
struct for which it does not have static knowledge, it cannot easily
extract the struct contents. The only solution is to use the
inconvenient DynAny interface.

Since Python is a dynamically typed language, it does not have this
difficulty. When omniORBpy receives an Any containing types it does
not know, it is able to create new Python types which behave exactly
as if there were statically generated stubs available. Note that this
behaviour is not required by the Python mapping specification, so
other Python ORBs may not be so accommodating.

The equivalent of DynAny creation can be achieved by dynamically
writing and importing new IDL, as described in
section~\ref{sec:importIDL}.

There is, however, a minor fly in the ointment when it comes to
receiving Anys. When an Any is transmitted, it is sent as a TypeCode
followed by the actual value.  Normally, the TypeCodes for entities
with names---members of structs, for example---contain those names as
strings. That permits omniORBpy to create types with the corresponding
names. Unfortunately, the GIOP specification permits TypeCodes to be
sent with empty strings where the names would normally
be\footnote{This is now deprecated, but some ORBs may still send empty
strings. No version of omniORB has ever sent empty strings.}. In this
situation, the types which omniORBpy creates cannot be given the
correct names. The contents of all types except structs and exceptions
can be accessed without having to know their names, through the
standard interfaces. Unknown structs and exceptions received by
omniORBpy have an attribute named `\code{\_values}' which contains a
sequence of the member values. This attribute is omniORBpy specific.

Similarly, TypeCodes for constructed types such as structs and unions
normally contain the repository ids of those types. This means that
omniORBpy can use types statically declared in the stubs when they are
available. Once again, the specification permits the repository id
strings to be empty\footnote{And once again, this practice is
deprecated.}. This means that even if stubs for a type received in an
Any are available, it may not be able to create a Python value with
the right type. For example, with a struct definition such as:

\begin{idllisting}
module M {
  struct S {
    string str;
    long   l;
  };
};
\end{idllisting}

\noindent The transmitted TypeCode for \type{M::S} may contain only
the information that it is a structure containing a string followed by
a long, not that it is type \type{M::S}, or what the member names are.

To cope with this situation, omniORBpy has an extension to the
standard interface which allows you to \term{coerce} an Any value to a
known type. Calling an Any's \op{value} method with a TypeCode
argument returns either a value of the requested type, or \code{None}
if the requested TypeCode is not \term{equivalent} to the Any's
TypeCode. The following code is guaranteed to be safe, but is not
standard:

\begin{pylisting}
a = # Acquire an Any from somewhere
v = a.value(CORBA.TypeCode(CORBA.id(M.S)))
if v is not None:
    print v.str
else:
    print "The Any does not contain a value compatible with M::S."
\end{pylisting}


\section{Interface Repository stubs}
\label{sec:ifrstubs}

The Interface Repository interfaces are declared in IDL module
\module{CORBA} so, according to the Python mapping, the stubs for them
should appear in the Python \module{CORBA} module, along with all the
other CORBA definitions. However, since the stubs are extremely large,
omniORBpy does not include them by default. To do so would
unnecessarily increase the memory footprint and start-up time.

The Interface Repository stubs are automatically included if you
define the \envvar{OMNIORBPY_IMPORT_IR_STUBS} environment variable.
Alternatively, you can import the stubs at run-time by calling the
\op{omniORB.importIRStubs} function. In both cases, the stubs become
available in the Python \module{CORBA} module.




\section{Using omniORBpy with omniORB 2.8}

omniORBpy is designed to work with omniORB 3. When it is used with
omniORB 2.8, many facilities are not available. This section describes
the limitations.

If you require facilities which are not available with omniORB 2.8,
you can cause your program to bail-out gracefully by detecting the
omniORB version at start-up. The \op{omniORB.coreVersion} function
returns a string of the form \textit{major.minor.micro} which
indicates what version of omniORB is in use. The two possible values
are currently `\texttt{2.8.0}' and `\texttt{3.0.0}'. Versions which
differ only in micro version number are guaranteed to be compatible
with each other.

\subsection{POA functions}

Under 2.8, only the root POA is available. It supports only the
following functions:

\begin{nsitemize}
\item \op{destroy}
\item \op{\_get\_the\_name}
\item \op{\_get\_the\_POAManager}
\item \op{activate\_object}
\item \op{deactivate\_object}
\item \op{servant\_to\_id}
\item \op{servant\_to\_reference}
\item \op{reference\_to\_servant}
\item \op{reference\_to\_id}
\item \op{id\_to\_servant}
\item \op{id\_to\_reference}
\end{nsitemize}

\noindent The \intf{POAManager} interface only supports:

\begin{nsitemize}
\item \op{activate}
\item \op{get\_state}
\end{nsitemize}

\noindent For both interfaces, all other operations fail, raising the
\code{CORBA.NO\_IMPLEMENT} exception.



\subsection{Local / remote transparency}

omniORB 3 goes to great lengths to make sure that the semantics of
local objects are identical to those for remote objects. omniORB 2.8
does not. omniORBpy also tries very hard to keep local and remote
semantics identical, but on the foundation of omniORB 2.8 it is not
always possible.

In most cases, you will not notice a difference between local and
remote operations. The cases where local/remote transparency is broken
under omniORB 2.8 are:

\begin{itemize}

\item An object is not deactivated until all local references to it
have been released.

\item When invoking on an object reference to a local object of the
wrong type, as described in section~\ref{sec:narrowing},
\code{CORBA.INV\_OBJREF} is not raised. Instead, if the operation name
does not exist on the object, \code{CORBA.NO\_IMPLEMENT} is raised; if
the operation name \emph{does} exist but the argument types are wrong,
\code{CORBA.BAD\_PARAM} is raised; if the operation exists and the
argument types are correct, the operation is executed.

\item Exception handlers (described later in
section~\ref{sec:exHandlers}) are not executed when local objects
raise system exceptions. Exceptions are always propagated to the
caller.

\end{itemize}

In all of these cases, omniORB 3 properly preserves local/remote
transparency.



%%%%%%%%%%%%%%%%%%%%%%%%%%%%%%%%%%%%%%%%%%%%%%%%%%%%%%%%%%%%%%%%%%%%%%
\chapter{Interoperable Naming Service}
%%%%%%%%%%%%%%%%%%%%%%%%%%%%%%%%%%%%%%%%%%%%%%%%%%%%%%%%%%%%%%%%%%%%%%
\label{chap:ins}

omniORB 3 supports the Interoperable Naming Service (INS), which will
be part of CORBA 2.4. The following is a summary of the new facilities
described in the INS edited chapters document~\cite{inschapters}.
These facilities are not available when using omniORBpy with omniORB
2.8.



\section{Object URIs}

As well as accepting IOR-format strings, \op{ORB::string\_to\_object}
now also supports two new Uniform Resource Identifier
(URI)~\cite{rfc2396} formats, which can be used to specify objects in
a convenient human-readable form. The existing IOR-format strings are
now also considered URIs.

\subsection{corbaloc}

\corbauri{corbaloc} URIs allow you to specify object references which
can be contacted by IIOP, or found through
\op{ORB::resolve\_initial\_references}. To specify an IIOP object
reference, you use a URI of the form:

\begin{quote}
\corbauri{corbaloc:iiop:}<\textit{host}>\corbauri{:}<\textit{port}>%
\corbauri{/}<\textit{object key}>
\end{quote}

\noindent for example:

\begin{quote}
\corbauri{corbaloc:iiop:myhost.example.com:1234/MyObjectKey}
\end{quote}

\noindent which specifies an object with key `MyObjectKey' within a
process running on myhost.example.com listening on port 1234. Object
keys containing non-ASCII characters can use the standard URI \%
escapes:

\begin{quote}
\corbauri{corbaloc:iiop:myhost.example.com:1234/My}%
\texttt{\%}%
\corbauri{efObjectKey}
\end{quote}

\noindent denotes an object key with the value 239 (hex ef) in the
third octet.

The protocol name `\corbauri{iiop}' can be abbreviated to the empty
string, so the original URI can be written:

\begin{quote}
\corbauri{corbaloc::myhost.example.com:1234/MyObjectKey}
\end{quote}

\noindent The IANA has assigned port number 2809\footnote{Not 2089 as
printed in \cite{inschapters}!} for use by \corbauri{corbaloc}, so if
the server is listening on that port, you can leave the port number
out.  The following two URIs refer to the same object:

\begin{quote}
\corbauri{corbaloc::myhost.example.com:2809/MyObjectKey}\\
\corbauri{corbaloc::myhost.example.com/MyObjectKey}
\end{quote}

\noindent You can specify an object which is available at more than
one location by separating the locations with commas:

\begin{quote}
\corbauri{corbaloc::myhost.example.com,:localhost:1234/MyObjectKey}
\end{quote}

\noindent Note that you must restate the protocol for each address,
hence the `\corbauri{:}' before `\corbauri{localhost}'. It could
equally have been written `\corbauri{iiop:localhost}'.

You can also specify an IIOP version number, although omniORB only
supports IIOP 1.0 at present:

\begin{quote}
\corbauri{corbaloc::1.2@myhost.example.com/MyObjectKey}
\end{quote}

\vspace{\baselineskip}

\noindent Alternatively, to use \op{resolve\_initial\_references}, you
use a URI of the form:

\begin{quote}
\corbauri{corbaloc:rir:/NameService}
\end{quote}


\subsection{corbaname}
\label{sec:corbaname}

\corbauri{corbaname} URIs cause \op{string\_to\_object} to look-up a
name in a CORBA Naming service. They are an extension of the
\corbauri{corbaloc} syntax:

\begin{quote}
\corbauri{corbaname:}%
<\textit{corbaloc location}>%
\corbauri{/}%
<\textit{object key}>%
\corbauri{#}%
<\textit{stringified name}>
\end{quote}

\noindent for example:

\begin{quote}
\corbauri{corbaname::myhost/NameService#project/example/echo.obj}\\
\corbauri{corbaname:rir:/NameService#project/example/echo.obj}
\end{quote}

\noindent Note that the object found with the
\corbauri{corbaloc}-style portion must be of type
\intf{CosNaming::NamingContext}, or something derived from it. If the
object key (or \corbauri{rir} name) is `\corbauri{NameService}', it
can be left out:

\begin{quote}
\corbauri{corbaname::myhost#project/example/echo.obj}\\
\corbauri{corbaname:rir:#project/example/echo.obj}
\end{quote}

\noindent The stringified name portion can also be left out, in which
case the URI denotes the \intf{CosNaming::NamingContext} which would
have been used for a look-up:

\begin{quote}
\corbauri{corbaname::myhost.example.com}\\
\corbauri{corbaname:rir:}
\end{quote}

\noindent The first of these examples is the easiest way of specifying
the location of a naming service.


\section{Configuring resolve\_initial\_references}
\label{sec:insargs}

The INS adds two new command line arguments which provide a portable
way of configuring \op{ORB::resolve\_initial\_references}:


\subsection{ORBInitRef}

\cmdline{-ORBInitRef} takes an argument of the form
<\textit{ObjectId}>\cmdline{=}<\textit{ObjectURI}>. So, for example,
with command line arguments of:

\begin{quote}
\cmdline{-ORBInitRef NameService=corbaname::myhost.example.com}
\end{quote}

\noindent \code{resolve\_initial\_references("NameService")} will
return a reference to the object with key `NameService' available on
myhost.example.com, port 2809. Since IOR-format strings are considered
URIs, you can also say things like:

\begin{quote}
\cmdline{-ORBInitRef NameService=IOR:00ff...}
\end{quote}


\subsection{ORBDefaultInitRef}

\cmdline{-ORBDefaultInitRef} provides a prefix string which is used to
resolve otherwise unknown names. When
\op{resolve\_initial\_references} is unable to resolve a name which
has been specifically configured (with \cmdline{-ORBInitRef}), it
constructs a string consisting of the default prefix, a `\corbauri{/}'
character, and the name requested.  The string is then fed to
\op{string\_to\_object}. So, for example, with a command line of:

\begin{quote}
\cmdline{-ORBDefaultInitRef corbaloc::myhost.example.com}
\end{quote}

\noindent a call to \code{resolve\_initial\_references("MyService")}
will return the object reference denoted by
`\corbauri{corbaloc::myhost.example.com/MyService}'.

Similarly, a \corbauri{corbaname} prefix can be used to cause
look-ups in the naming service. Note, however, that since a
`\corbauri{/}' character is always added to the prefix, it is
impossible to specify a look-up in the root context of the naming
service---you have to use a sub-context, like:

\begin{quote}
\cmdline{-ORBDefaultInitRef corbaname::myhost.example.com\#services}
\end{quote}


\subsection{omniORB configuration file}

As an extension to the standard facilities of the INS, omniORB
supports configuration file entries named \texttt{ORBInitRef} and
\texttt{ORBDefaultInitRef}. The syntax is identical to the command
line arguments. \file{omniORB.cfg} might contain:

\begin{quote}
\begin{verbatim}
ORBInitRef NameService=corbaname::myhost.example.com
ORBDefaultInitRef corbaname:rir:#services
\end{verbatim}
\end{quote}


\subsection{Resolution order}

With all these options for specifying object references to be returned
by \op{resolve\_\dsc{}initial\_references}, it is important to
understand the order in which the options are tried. The resolution
order, as required by the CORBA specification, is:

\begin{enumerate}

\item Check for special names such as `\texttt{RootPOA}'\footnote{In
fact, a strict reading of the specification says that it should be
possible to override `\texttt{RootPOA}' etc.\ with
\cmdline{-ORBInitRef}, but since POAs are locality constrained that is
ridiculous.}.

\item Resolve with an \cmdline{-ORBInitRef} argument.

\item \label{itm:defaultgotcha} Resolve with the
\cmdline{-ORBDefaultInitRef} prefix, if present.

\item Resolve with an \texttt{ORBInitRef}
(or old-style \texttt{NAMESERVICE}) entry in the configuration file.

\item Resolve with the \texttt{ORBDefaultInitRef} entry in the
configuration file, if present.

\item Resolve with the deprecated \texttt{ORBInitialHost} boot agent.

\end{enumerate}


\noindent This order mostly has the expected consequences---in
particular that command line arguments override entries in the
configuration file.  However, you must be careful with the default
prefixes. Suppose you have configured a `\texttt{NameService}' entry
in the configuration file, and you specify a default prefix on the
command line with:

\begin{quote}
\cmdline{-ORBDefaultInitRef corbaname:rir:\#services}
\end{quote}

\noindent expecting unknown services to be looked up in the configured
naming service. Now, step~\ref{itm:defaultgotcha} above means that
\code{resolve\_initial\_references("MyService")} should be processed
with the steps:

\begin{enumerate}

\item Construct the URI `\corbauri{corbaname:rir:#services/MyService}'
and give it to \op{string\_to\_object}.

\item Resolve the first part of the \corbauri{corbaname} URI by
calling\\\code{resolve\_initial\_references("NameService")}.

\item Construct the URI
`\corbauri{corbaname:rir:#services/NameService}' and give it to
\op{string\_to\_object}.

\item Resolve the first part of the \corbauri{corbaname} URI by
calling\\\code{resolve\_initial\_references("NameService")}.

\item \dots\ and so on for ever\dots

\end{enumerate}


\noindent omniORB detects loops like this and throws either
\code{CORBA.ORB.InvalidName} if the loop started with a call to
\op{resolve\_initial\_references}, or \code{CORBA.\dsc{}BAD\_PARAM}
if it started with a call to \op{string\_to\_object}. To avoid the
problem you must either specify the \texttt{NameService} reference on
the command line, or put the \texttt{DefaultInitRef} in the
configuration file.



\section{omniNames}

\subsection{NamingContextExt}

omniNames now supports the \intf{CosNaming::NamingContextExt}
interface:

\begin{idllisting}
module CosNaming {
  interface NamingContextExt : NamingContext {
    typedef string StringName;
    typedef string Address;
    typedef string URLString;

    StringName  to_string(in Name n)        raises(InvalidName);
    Name        to_name  (in StringName sn) raises(InvalidName);

    exception InvalidAddress {};

    URLString   to_url(in Address addr, in StringName sn)
      raises(InvalidAddress, InvalidName);

    Object      resolve_str(in StringName n)
      raises(NotFound, CannotProceed, InvalidName, AlreadyBound);
  };
};
\end{idllisting}

\op{to\_string} and \op{to\_name} convert from \type{CosNaming::Name}
sequences to flattened strings and vice-versa.  Note that calling
these operations involves remote calls to the naming service, so they
are not particularly efficient. The \module{omniORB.URI} module
contains equivalent \op{nameToString} and \op{stringToName} functions,
which do not involve remote calls.

A \type{CosNaming::Name} is stringified by separating name components
with `\texttt{/}' characters. The \code{kind} and \code{id} fields of
each component are separated by `\texttt{.}' characters. If the
\code{kind} field is empty, the representation has no trailing
`\texttt{.}'; if the \code{id} is empty, the representation starts
with a `\texttt{.}' character; if both \texttt{id} and \texttt{kind}
are empty, the representation is just a `\texttt{.}'. The backslash
`\texttt{\textbackslash}' is used to escape the meaning of
`\texttt{/}', `\texttt{.}' and `\texttt{\textbackslash}' itself.

\op{to\_url} takes a \corbauri{corbaloc} style address and key string
(but without the \corbauri{corbaloc:} part), and a stringified name,
and returns a \corbauri{corbaname} URI (incorrectly called a URL)
string, having properly escaped any invalid characters. The
specification does not make it clear whether or not the address string
should also be escaped by the operation; omniORB does not escape
it. For this reason, it is best to avoid calling \op{to\_url} if the
address part contains escapable characters.  The local function
\op{omniORB.URI.addrAndNameToURI} is equivalent.

\op{resolve\_str} is equivalent to calling \op{to\_name} followed by
the inherited \op{resolve} operation. There are no string-based
equivalents of the various bind operations.


\subsection{Use with corbaname}

To make it easy to use omniNames with \corbauri{corbaname} URIs, it
now starts with the default port of 2809, and an object key of
`\texttt{NameService}' for the root naming context. This is only
possible when it is started `fresh', rather than with a log file from
an older omniNames version.

If you have a previous omniNames log, configured to run on a different
port, and with a different object key for its root context, all is not
lost. If the root context's object key is not `\texttt{NameService}',
omniNames creates a forwarding agent with that key. Effectively, this
means that there are two object keys which refer to the root
context---`\texttt{NameService}' and whatever the original key was.

For the port number, there are two options. The first is to run
omniNames with a command line argument like:

\begin{quote}
\cmdline{omniNames -logdir /the/log/dir -ORBpoa\_iiop\_port 2809}
\end{quote}

\noindent This causes it to listen on both port 2809 \emph{and}
whatever port it listened on before. The disadvantage with this is
that the IORs to all naming contexts now contain two IIOP profiles,
one for each port, which, amongst other things, increases the size of
the omniNames log.

The second option is to use omniMapper, as described below.


\section{omniMapper}
\hyphenation{omni-Mapper}

omniMapper is a simple daemon which listens on port 2809 (or any other
port), and redirects IIOP requests for configured object keys to
associated persistent IORs. It can be used to make a naming service
(even an old non-INS aware version of omniNames or other ORB's naming
service) appear on port 2809 with the object key
`\texttt{NameService}'. The same goes for any other service you may
wish to specify, such as an interface repository. omniMapper is
started with a command line of:

\begin{quote}
\cmdline{omniMapper [-port }<\textit{port}>%
\cmdline{] [-config }<\textit{config file}>%
\cmdline{] [-v]}
\end{quote}

\noindent The \cmdline{-port} option allows you to choose a port other
than 2809 to listen on\footnote{You can also play the
\cmdline{-ORBpoa\_iiop\_port} trick to make it listen on more than one
port.}. The \cmdline{-config} option specifies a location for the
configuration file. The default name is \file{/etc/omniMapper.cfg}, or
%BEGIN LATEX
\file{C:\omniMapper.cfg}
%END LATEX
%HEVEA\verb|C:\omniMapper.cfg|
on Windows. omniMapper does not normally print anything; the
\cmdline{-v} option makes it verbose so it prints configuration
information and a record of the redirections it makes, to standard
output.

The configuration file is very simple. Each line contains a string to
be used as an object key, some white space, and an IOR (or any valid
URI) that it will redirect that object key to. Comments should be
prefixed with a `\texttt{\#}' character. For example:

\begin{quote}
\begin{verbatim}
# Example omniMapper.cfg
NameService         IOR:000f...
InterfaceRepository IOR:0100...
\end{verbatim}
\end{quote}

omniMapper can either be run on a single machine, in much the same way
as omniNames, or it can be run on \emph{every} machine, with a common
configuration file. That way, each machine's omniORB configuration
file could contain the line:

\begin{quote}
\begin{verbatim}
ORBDefaultInitRef corbaloc::localhost
\end{verbatim}
\end{quote}



\section{Creating objects with simple object keys}

In normal use, omniORB creates object keys containing various
information including POA names and various non-ASCII characters.
Since object keys are supposed to be opaque, this is not usually a
problem. The INS breaks this opacity and requires servers to create
objects with human-friendly keys.

If you wish to make your objects available with human-friendly URIs,
there are two options. The first is to use omniMapper as described
above, in conjunction with a \code{PERSISTENT} POA. The second is to
create objects with the required keys yourself. You do this with a
special POA with the name `\texttt{omniINSPOA}', acquired from
\op{resolve\_initial\_references}. This POA has the \code{USER\_ID}
and \code{PERSISTENT} policies, and the special property that the
object keys it creates contain only the object ids given to the POA,
and no other data. It is a normal POA in all other respects, so you
can activate/deactivate it, create children, and so on, in the usual
way.




%%%%%%%%%%%%%%%%%%%%%%%%%%%%%%%%%%%%%%%%%%%%%%%%%%%%%%%%%%%%%%%%%%%%%%
\chapter{The IDL compiler}
%%%%%%%%%%%%%%%%%%%%%%%%%%%%%%%%%%%%%%%%%%%%%%%%%%%%%%%%%%%%%%%%%%%%%%
\label{chap:omniidl}

omniORBpy uses a new IDL compiler, named omniidl, which is also used
by omniORB 3. It consists of a generic front-end parser written in
C++, and a number of back-ends written in Python. omniidl is very
strict about IDL validity, so you may find that it reports errors in
IDL which compiles fine with earlier versions of omniORB, and with
other ORBs.

The general form of an omniidl command line is:

\begin{quote} % Not the clearest bit of mark-up ever... :-)
\cmdline{omniidl }[\textit{options}]\cmdline{ -b}%
<\textit{back-end}>\cmdline{ }[\textit{back-end options}]%
\cmdline{ }<\textit{file 1}>\cmdline{ }<\textit{file 2}>%
\cmdline{ }\dots
\end{quote}

\section{Common options}

The following options are common to all back-ends:

\begin{tabbing}
\cmdline{-D}\textit{name}[\cmdline{=}\textit{value}]~~ \= \kill
%HEVEA\\

\cmdline{-D}\textit{name}[\cmdline{=}\textit{value}]
     \> Define \textit{name} for the preprocessor.\\

\cmdline{-U}\textit{name}
     \> Undefine \textit{name} for the preprocessor.\\

\cmdline{-I}\textit{dir}
     \> Include \textit{dir} in the preprocessor search path.\\

\cmdline{-E}
     \> Only run the preprocessor, sending its output to stdout.\\

\cmdline{-Y}\textit{cmd}
     \> Use \textit{cmd} as the preprocessor, rather than the normal C
        preprocessor.\\

\cmdline{-N}
     \> Do not run the preprocessor.\\

\cmdline{-Wp}\textit{arg}[,\textit{arg}\dots]
     \> Send arguments to the preprocessor.\\

\cmdline{-b}\textit{back-end}
     \> Run the specified back-end. For omniORBpy, use \cmdline{-bpython}.\\

\cmdline{-Wb}\textit{arg}[,\textit{arg}\dots]
     \> Send arguments to the back-end.\\

\cmdline{-nf}
     \> Do not warn about unresolved forward declarations.\\

\cmdline{-k}
     \> Keep comments after declarations, to be used by some back-ends.\\

\cmdline{-K}
     \> Keep comments before declarations, to be used by some back-ends.\\

\cmdline{-C}\textit{dir}
     \> Change directory to \textit{dir} before writing output files.\\

\cmdline{-d}
     \> Dump the parsed IDL then exit, without running a back-end.\\

\cmdline{-p}\textit{dir}
     \> Use \textit{dir} as a path to find omniidl back-ends.\\

\cmdline{-V}
     \> Print version information then exit.\\

\cmdline{-u}
     \> Print usage information.\\

\cmdline{-v}
     \> Verbose: trace compilation stages.\\

\end{tabbing}

\noindent Most of these options are self explanatory, but some are not
so obvious.

\subsection{Preprocessor interactions}

IDL is processed by the C preprocessor before omniidl parses it.
Unlike the old IDL compiler, which used different C preprocessors on
different platforms, omniidl always uses the GNU C preprocessor (which
it builds with the name omnicpp). The \cmdline{-D}, \cmdline{-U}, and
\cmdline{-I} options are just sent to the preprocessor. Note that the
current directory is not on the include search path by default---use
`\cmdline{-I.}' for that. The \cmdline{-Y} option can be used to
specify a different preprocessor to omnicpp. Beware that line
directives inserted by other preprocessors are likely to confuse
omniidl.


\subsection{Forward-declared interfaces}

If you have an IDL file like:

\begin{idllisting}
interface I;
interface J {
  attribute I the_I;
};
\end{idllisting}

\noindent then omniidl will normally issue a warning:

{\small
\begin{verbatim}
  test.idl:1: Warning: Forward declared interface `::I' was never
  fully defined
\end{verbatim}
}

\noindent It is illegal to declare such IDL in isolation, but it
\emph{is} valid to define interface \intf{I} in a separate file. If
you have a lot of IDL with this sort of construct, you will drown
under the warning messages. Use the \cmdline{-nf} option to suppress
them.


\subsection{Comments}

By default, omniidl discards comments in the input IDL. However, with
the \cmdline{-k} and \cmdline{-K} options, it preserves the comments
for use by the back-ends. The C++ back-end ignores this information,
but it is relatively easy to write new back-ends which \emph{do} make
use of comments.

The two different options relate to how comments are attached to
declarations within the IDL. Given IDL like:

\begin{idllisting}
interface I {
  void op1();
  // A comment
  void op2();
};
\end{idllisting}

\noindent the \cmdline{-k} flag will attach the comment to \op{op1};
the \cmdline{-K} flag will attach it to \op{op2}.



\section{Python back-end options}
\label{sec:Wbglobal}

When you specify the Python back-end (with \cmdline{-bpython}), the
following \cmdline{-Wb} options are available. Note that the
\cmdline{-Wb} options must be specified \emph{after} the
\cmdline{-bpython} option, so omniidl knows which back-end to give the
arguments to.


\begin{tabbing}

\cmdline{-Wbmodules=}\textit{p}~~ \= \kill
%HEVEA\\

\cmdline{-Wbstdout}
     \> Send the generated stubs to standard output, rather than to a
        file.\\

\cmdline{-Wbinline}
     \> Output stubs for \#included files in line with the main
        file.\\

\cmdline{-Wbglobal=}\textit{g}
     \> Use \textit{g} as the name for the global IDL scope (default
        \module{\_GlobalIDL}).\\

\cmdline{-Wbpackage=}\textit{p}
     \> Put both Python modules and stub files in package
        \textit{p}.\\

\cmdline{-Wbmodules=}\textit{p}
     \> Put Python modules in package \textit{p}.\\

\cmdline{-Wbstubs=}\textit{p}
     \> Put stub files in package \textit{p}.\\

\end{tabbing}


The \cmdline{-Wbstdout} option is not really useful if you are
invoking omniidl yourself. It is used by \op{omniORB.importIDL},
described in section~\ref{sec:importIDL}.

When you compile an IDL file which \#includes other IDL files, omniidl
normally only generates code for the main file, assuming that code for
the included files will be generated separately. Instead, you can use
the \cmdline{-Wbinline} option to generate code for the main IDL file
\emph{and} all included files in a single stub file.

Definitions declared at IDL global scope are normally placed in a
Python module named `\module{\_GlobalIDL}'. The \cmdline{-Wbglobal}
allows you to change that.


As explained in section~\ref{sec:generatingStubs}, when you compile an
IDL file like:

\begin{idllisting}
// echo_example.idl
module Example {
  interface Echo {
    string echoString(in string mesg);
  };
};
\end{idllisting}

\noindent omniidl generates directories named \file{Example} and
\file{Example__POA}, which provide the standard Python mapping
modules, and also the file \file{example_echo_idl.py} which contains
the actual definitions. The latter file contains code which inserts
the definitions in the standard modules. This arrangement means that
it is not possible to move all of the generated code into a Python
package by simply placing the files in a suitably named directory.
You may wish to do this to avoid clashes with names in use elsewhere
in your software.

You can place all generated code in a package using the
\cmdline{-Wbpackage} command line option. For example,

\begin{quote}
\cmdline{omniidl -bpython -Wbpackage=generated echo\_example.idl}
\end{quote}

\noindent creates a directory named `\file{generated}', containing the
generated code. The stub module is now called
`\module{generated.Example}', and the actual stub definitions are in
`\module{generated.example\_echo\_idl}'. If you wish to split the
modules and the stub definitions into different Python packages, you
can use the \cmdline{-Wbmodules} and \cmdline{-Wbstubs} options.

Note that if you use these options to change the module package, the
interface to the generated code is not strictly-speaking CORBA
compliant. You may have to change your code if you ever use a Python
ORB other than omniORBpy.



\section{Examples}

Generate the Python stubs for a file \file{a.idl}:

\begin{quote}
\cmdline{omniidl -bpython a.idl}
\end{quote}

\noindent As above, but put the stubs in a package called
`\module{stubs}':

\begin{quote}
\cmdline{omniidl -bpython -Wbstubs=stubs a.idl}
\end{quote}

\noindent Generate both Python and C++ stubs for two IDL files
(requires omniORB 3):

\begin{quote}
\cmdline{omniidl -bpython -bcxx a.idl b.idl}
\end{quote}

\noindent Just check the IDL files for validity, generating no output:

\begin{quote}
\cmdline{omniidl a.idl b.idl}
\end{quote}




%%%%%%%%%%%%%%%%%%%%%%%%%%%%%%%%%%%%%%%%%%%%%%%%%%%%%%%%%%%%%%%%%%%%%%
\chapter{The omniORBpy API}
%%%%%%%%%%%%%%%%%%%%%%%%%%%%%%%%%%%%%%%%%%%%%%%%%%%%%%%%%%%%%%%%%%%%%%
\label{omniORBapi}

In this chapter, we introduce the omniORBpy API. The purpose of this
API is to provide access points to omniORB specific functionality that
is not covered by the CORBA specification.  Obviously, if you use this
API in your application, that part of your code is not going to be
portable to run unchanged on other vendors' ORBs. To make it easier to
identify omniORB dependent code, this API is defined in the
`\module{omniORB}' module.



\section{ORB initialisation options}
\label{omniorbapioptions}\label{sec:ORBargs}

\op{CORBA::ORB\_init} accepts the following standard command-line
arguments:

\begin{tabbing}
\cmdline{-ORBclientCallTimeOutPeriod }<\textit{0--max integer}> \=\kill
%HEVEA\\

\cmdline{-ORBid omniORB3} \> The identifier must be `omniORB3'.\\

\cmdline{-ORBInitRef }<\textit{ObjectId}>=<\textit{ObjectURI}>
                          \> See section~\ref{sec:insargs}.\\

\cmdline{-ORBDefaultInitRef }<\textit{Default URI}>
                          \> See section~\ref{sec:insargs}.
\end{tabbing}


\noindent and the following omniORB-specific arguments:

\begin{tabbing}
\cmdline{-ORBclientCallTimeOutPeriod }<\textit{0--max integer}> \=\kill
%HEVEA\\

\cmdline{-ORBtraceLevel }<\textit{level}>\> See section~\ref{sec:rttrace}.\\

\cmdline{-ORBtraceInvocations}           \> See section~\ref{sec:rttrace}.\\

\cmdline{-ORBstrictIIOP}                 \> See section~\ref{sec:strictIIOP}.\\

\cmdline{-ORBgiopMaxMsgSize }<\textit{size in bytes}>
                                         \> See section~\ref{sec:giopmsg}.\\

\cmdline{-ORBobjectTableSize }<\textit{number of entries}>
                                         \> See section~\ref{sec:objtable}.\\

\cmdline{-ORBserverName }<\textit{string}>
                                         \> See section~\ref{sec:servername}.\\

\cmdline{-ORBno\_bootstrap\_agent}       \> See section~\ref{sec:bootstrap}.\\

\cmdline{-ORBverifyObjectExistsAndType }<\textit{0 or 1}>
                                         \> See section~\ref{sec:lcd}.\\

\cmdline{-ORBinConScanPeriod }<\textit{0--max integer}>
                                         \> See section~\ref{sec:shut}.\\

\cmdline{-ORBoutConScanPeriod }<\textit{0--max integer}>
                                         \> See section~\ref{sec:shut}.\\

\cmdline{-ORBclientCallTimeOutPeriod }<\textit{0--max integer}>
                                         \> See section~\ref{sec:shut}.\\

\cmdline{-ORBserverCallTimeOutPeriod }<\textit{0--max integer}>
                                         \> See section~\ref{sec:shut}.\\

\cmdline{-ORBscanGranularity }<\textit{0--max integer}>
                                         \> See section~\ref{sec:shut}.\\

\cmdline{-ORBlcdMode}                    \> See section~\ref{sec:lcd}.\\

\cmdline{-ORBpoa\_iiop\_port }<\textit{port no.}>
                                         \> See section~\ref{sec:nameport}.\\

\cmdline{-ORBpoa\_iiop\_name\_port }<\textit{hostname}[\textit{:port no.]}>
                                         \> See section~\ref{sec:nameport}.\\

\cmdline{-ORBhelp}                     \> Lists all ORB command line options.

\end{tabbing}

\noindent and these two obsolete omniORB-specific arguments:

\begin{tabbing}
\cmdline{-ORBclientCallTimeOutPeriod }<\textit{0--max integer}> \=\kill
%HEVEA\\

\cmdline{-ORBInitialHost }<\textit{string}>
                                         \> See section~\ref{sec:bootstrap}.\\

\cmdline{-ORBInitialPort }<\textit{1--65535}>]
                                         \> See section~\ref{sec:bootstrap}.

\end{tabbing}


\noindent As defined in the CORBA specification, any command-line
arguments understood by the ORB will be removed from \code{argv} when
the initialisation functions return. Therefore, an application is not
required to handle any command-line arguments it does not understand.



\section{Hostname and port}
\label{sec:nameport}

Normally, omniORB lets the operating system pick which port number it
should use to listen for IIOP calls. Alternatively, you can specify a
particular port using \cmdline{-ORBpoa\_iiop\_port}. If you specify
\cmdline{-ORBpoa\_iiop\_port} more than once, omniORB will listen on
all the ports you specify.

By default, the ORB can work out the IP address of the host machine.
This address is recorded in the object references of the local
objects. However, when the host has multiple network interfaces and
multiple IP addresses, it may be desirable for the application to
control what address the ORB should use. This can be done by defining
the environment variable \code{OMNIORB\_USEHOSTNAME} to contain the
preferred host name or IP address in dot-numeric form. Alternatively,
the same can be achieved using the \cmdline{-ORBpoa\_iiop\_name\_port}
option. You can optionally specify a port number too. Again, you can
specify more than one host name by using
\cmdline{-ORBpoa\_iiop\_name\_port} more than once.

When using omniORB 2.8, these two options are named
\cmdline{-BOAiiop\_port} and \cmdline{-BOAiiop\_name\_port}.



\section{Run-time Tracing and Diagnostic Messages}
\label{sec:rttrace}

omniORB can output tracing and diagnostic messages to the standard
error stream. You can vary the amount of tracing using the
\cmdline{-ORBtraceLevel }<\textit{level}> command line argument. For
instance:

\begin{verbatim}
$ example_echo_srv.py -ORBtraceLevel 5
\end{verbatim}

%$ To fix Emacs' syntax highlighting

\noindent At the moment, the following trace levels are defined:

\vspace{\baselineskip}

\begin{tabular}{lp{.6\textwidth}}
%HEVEA\\

level 0      & turn off all tracing and informational messages\\
level 1      & informational messages only\\
level 2      & the above plus configuration information\\

level 5      & the above plus notifications when server threads are
               created or communication endpoints are shutdown\\

level 10--20 & the above plus execution and exception traces\\

level 25     & the above plus hex dumps of all data sent and received
               by the ORB via its network connections.\\
\end{tabular}

\vspace{\baselineskip}

\noindent With omniORB 3, you can also use
\cmdline{-ORBtraceInvocations} to trace all operation invocations.


\section{Server Name}
\label{sec:servername}

Applications can optionally specify a name to identify the server
process. At the moment, this name is only used by the host-based
access control module. See section~\ref{sec:accept} for details. The
server name can be changed by specifying the command-line option:
\cmdline{-ORBserverName }<\textit{string}>.


\section{GIOP Message Size}
\label{sec:giopmsg}

omniORB sets a limit on the GIOP message size that can be sent or
received. The maximum size can be set with the command-line option
\cmdline{-ORBgiopMaxMsgSize}.  The exact value is somewhat
arbitrary. The reason such a limit exists is to provide some way to
protect the server side from resource exhaustion. Think about the case
when the server receives a rogue GIOP(IIOP) request message that
contains a sequence length field set to 2$^{31}$. With a reasonable
message size limit, the server can reject this rogue message straight
away.


\section{Object table size}
\label{sec:objtable}

omniORB uses a hash table to store the mapping from object keys to
servant objects. Normally, it dynamically re-sizes the hash table when
it becomes too full or too empty. This is the most efficient trade-off
between performance and memory usage. However, since all POA
operations which add or remove objects from the table can (very
occasionally) cause the object table to resize, the time spent in POA
operations is much less predictable than if the table size was fixed.

The \cmdline{-ORBobjectTableSize} argument allows you to choose a
fixed size for the object table. This prevents omniORB from resizing
it. Note that omniORB uses an open hash table so you can have any
number of objects active, no matter what size table you specify. If
you have many more active objects than hash table entries, object
look-up performance will become linear with the number of objects.



\section{Obsolete Initial Object Reference Bootstrapping}
\label{sec:bootstrap}

Starting from 2.6.0, but superseded by the Interoperable Naming
Service in omniORB 3, a mechanism is available for the ORB runtime to
obtain the initial object references to CORBA services. The bootstrap
service is a special object with the object key `INIT' and the
following interface\footnote{This interface was first defined by Sun's
NEO and is in used in Sun's JavaIDL.}:

\begin{idllisting}
  // IDL
  module CORBA {
    interface InitialReferences {
      Object get(in ORB::ObjectId id);
      // returns the initial object reference of the service
      // identified by <id>. For example the id for the
      // Naming service is "NameService".

      ORB::ObjectIdList list();
      // returns the list of service ids that this agent knows
   };
 };
\end{idllisting}

By default, all omniORB servers contain an instance of this object and
are able to respond to remote invocations. To prevent the ORB from
instantiating this object, the command-line option
\cmdline{-ORBno\_bootstrap\_agent} should be specified.

In particular, the Naming Service omniNames is able to respond to a
query through this interface and return the object reference of its
root context. In effect, the bootstrap agent provides a level of
indirection.  All omniORB clients still have to be supplied with the
address of the bootstrap agent. However, the information is much
easier to specify than a stringified IOR!  Another advantage of this
approach is that it is completely compatible with JavaIDL. This makes
it possible for programs written for JavaIDL to share a Naming
Service with omniORB.

The address of the bootstrap agent is given by the
\texttt{ORBInitialHost} and \texttt{ORBInitialPort} parameter in the
omniORB configuration file (section~\ref{sec:setup}). The parameters
can also be specified as command-line options
(section~\ref{omniorbapioptions}). The parameter
\texttt{ORBInitialPort} is optional. If it is not specified, port
number 900 will be used.

During initialisation, the ORB reads the parameters in the omniORB
configuration file. If the parameter \texttt{NAMESERVICE} is
specified, the stringified IOR is used as the object reference of the
root naming context. If the parameter is absent and the parameter
\texttt{ORBInitialHost} is present, the ORB contacts the bootstrap
agent at the address specified to obtain the root naming context when
the application calls \op{resolve\_initial\_references}. If neither is
present, \op{resolve\_\dsc{}initial\_references} returns a nil object
reference. Finally, the command line argument
\cmdline{-ORBInitialHost} overrides any parameters in the
configuration file. The ORB always contacts the bootstrap agent at the
address specified to obtain the root naming context.

Now we are ready to describe a simple way to set up omniNames.

\begin{enumerate}
\item Start omniNames for the first time on a machine (e.g. wobble):

{\tt \$ omniNames -start 1234}

\item Add to omniORB.cfg:

{\tt ORBInitialHost wobble}

{\tt ORBInitialPort 1234}

\item All omniORB applications will now be able to contact omniNames.

Alternatively, the command line options can be used, for example:

{\tt \$ eg3\_impl -ORBInitialHost wobble -ORBInitialPort 1234 \&}

{\tt \$ eg3\_clt -ORBInitialHost wobble -ORBInitialPort 1234}

\end{enumerate}




\section{GIOP Lowest Common Denominator Mode}
\label{sec:lcd}
\label{sec:strictIIOP}

Sometimes, to cope with bugs in another ORB, it is necessary to
disable various GIOP and IIOP features in order to achieve
interoperability. If the command line option \cmdline{-ORBlcdMode} is
present, the ORB enters the so-called `lowest common denominator
mode'. It bends over backwards to cope with bugs in the ORB at the
other end. This is purely a transitional measure. The long term
solution is to report the bugs to the other vendors and ask them to
fix them expediently.

In some (sloppy) IIOP implementations, the message size value in the
IIOP header can be larger than the actual body size, i.e.\ there is
garbage at the end. As the spec does not say the message size must
match the body size exactly, this is not a clear violation of the
spec. omniORB's default policy is to expect incoming messages to be
formatted properly. Any messages that have garbage at the end will be
rejected.

\cmdline{-ORBlcdMode} sets omniORB to silently skip the unread part of
such invalid messages. Alternatively, you can change just this policy
with a command line argument of \cmdline{-ORBstrictIIOP 0}. The
problem with doing this is that the header message size may actually
be garbage, caused by a bug in the sender's code. The receiving thread
may block forever as it tries to read more data from the
connection. In this case the sender won't send any more as it thinks
it has marshalled in all the data.

By default, omniORB uses the GIOP LOCATE\_REQUEST message to verify
the existence of an object prior to the first invocation. If another
vendor's ORB is known not to be able to handle this GIOP message, you
can disable this feature with the
\cmdline{-ORBverifyObjectExistsAndType} option, and hence achieve
interoperability.




\section{System Exception Handlers}
\label{sec:exHandlers}

By default, all system exceptions which are raised during an operation
invocation, with the exception of \code{CORBA.TRANSIENT}, are
propagated to the application code. Some applications may prefer to
trap these exceptions within the proxy objects so that the application
logic does not have to deal with the error condition. For example,
when a \code{CORBA.COMM\_FAILURE} is received, an application may just
want to retry the invocation until it finally succeeds. This approach
is useful for objects that are persistent and their operations are
idempotent.

omniORBpy provides a set of functions to install exception handlers.
Once they are installed, proxy objects will call these handlers when
the associated system exceptions are raised by the ORB runtime.
Handlers can be installed for \code{CORBA.\dsc{}TRANSIENT},
\code{CORBA.COMM\_FAILURE} and \code{CORBA.SystemException}.  This
last handler covers all system exceptions other than the two covered
by the first two handlers. An exception handler can be installed for
individual proxy objects, or it can be installed for all proxy objects
in the address space.


\subsection{CORBA.TRANSIENT handlers}

When a \code{CORBA.TRANSIENT} exception is raised by the ORB runtime,
the default behaviour of the proxy objects is to retry indefinitely
until the operation succeeds, with an exponentially increasing delay
(up to a limit) between retries.

The ORB runtime will raise \code{CORBA.TRANSIENT} under the following
conditions:

\begin{enumerate}

\item When a \emph{cached} network connection is broken while an
operation invocation is in progress. The ORB will try to open a new
connection at the next invocation.

\item When the proxy object has been redirected by a location forward
message by the remote object to a new location and the object at the
new location cannot be contacted. In addition to the
\code{CORBA.TRANSIENT} exception, the proxy object also resets its
internal state so that the next invocation will be directed at the
original location of the remote object.

\item When the remote object reports \code{CORBA.TRANSIENT}.

\end{enumerate}

You can override the default behaviour by installing your own
exception handler. The function to call has signature:

\begin{pylisting}
omniORB.installTransientExceptionHandler(cookie, function [, object])
\end{pylisting}

The arguments are a cookie, which is any Python object, a call-back
function, and optionally an object reference. If the object reference
is present, the exception handler is installed for just that object;
otherwise the handler is installed for all objects with no handler of
their own.

The call-back function must have the signature

\begin{pylisting}
function(cookie, retries, exc) -> boolean
\end{pylisting}

When a \code{TRANSIENT} exception occurs, the function is called,
passing the cookie object, a count of how many times the operation has
been retried, and the TRANSIENT exception object itself. If the
function returns true, the operation is retried; if it returns false,
the TRANSIENT exception is raised in the application.



\subsection{CORBA.COMM\_FAILURE and CORBA.SystemException}

There are two other functions for registering exception handlers: one
for \code{CORBA.\dsc{}COMM\_FAILURE}, and one for all other
exceptions. For both these cases, the default is for there to be no
handler, so exceptions are propagated to the application.

\begin{pylisting}
omniORB.installCommFailureExceptionHandler(cookie, function [, object])
omniORB.installSystemExceptionHandler(cookie, function [, object])
\end{pylisting}

\noindent In both cases, the call-back function has the same signature
as for \code{TRANSIENT} handlers.




\section{Dynamic importing of IDL}
\label{sec:importIDL}

omniORBpy is usually used with pre-generated stubs. Since Python is a
dynamic language, however, it is possible to compile and import new
stubs at run-time.

Dynamic importing is achieved with \op{omniORB.importIDL} and
\op{omniORB.\dsc{}importIDLString}. Their signatures are:

\begin{pylisting}
importIDL(filename [, args ]) -> tuple
importIDLString(string [, args ]) -> tuple
\end{pylisting}

The first function compiles and imports the specified file; the second
takes a string containing the IDL definitions. The functions work by
forking omniidl and importing its output; they both take an optional
argument containing a list of strings which are used as arguments for
omniidl. For example, the following command runs omniidl with an
include path set:

\begin{pylisting}
m = omniORB.importIDL("test.idl", ["-I/my/include/path"])
\end{pylisting}

\noindent Instead of specifying omniidl arguments on each import, you
can set the arguments to be used for all calls using the
\op{omniORB.omniidlArguments} function.

Both import functions return a tuple containing the names of the
Python modules which have been imported. The modules themselves can be
accessed through \code{sys.modules}. For example:

\begin{idllisting}
// test.idl
const string s = "Hello";
module M1 {
  module M2 {
    const long l = 42;
  };
};
module M3 {
  const short s = 5;
};
\end{idllisting}

\noindent From Python:

\begin{pylisting}
>>> import sys, omniORB
>>> omniORB.importIDL("test.idl")
('M1', 'M1.M2', 'M3', '_GlobalIDL')
>>> sys.modules["M1.M2"].l
42
>>> sys.modules["M3"].s
5
>>> sys.modules["_GlobalIDL"].s
'Hello'
\end{pylisting}



%%%%%%%%%%%%%%%%%%%%%%%%%%%%%%%%%%%%%%%%%%%%%%%%%%%%%%%%%%%%%%%%%%%%%%
\chapter{Connection Management}
%%%%%%%%%%%%%%%%%%%%%%%%%%%%%%%%%%%%%%%%%%%%%%%%%%%%%%%%%%%%%%%%%%%%%%
\label{ch_conn}


This chapter describes how omniORB manages network connections.

\section{Background}

In CORBA, the ORB is the `middleware' that allows a client to invoke
an operation on an object without regard to its implementation or
location. In order to invoke an operation on an object, a client needs
to `bind' to the object by acquiring its object reference. Such a
reference may be obtained as the result of an operation on another
object (such as a naming service) or by conversion from a stringified
representation. If the object is in a different address space, the
binding process involves the ORB building a proxy object in the
client's address space. The ORB arranges for invocations on the proxy
object to be transparently mapped to equivalent invocations on the
implementation object.

For the sake of interoperability, CORBA mandates that all ORBs should
support IIOP as the means to communicate remote invocations over a
TCP/IP connection. IIOP is asymmetric with respect to the roles of the
parties at the two ends of a connection. At one end is the client
which can only initiate remote invocations. At the other end is the
server which can only receive remote invocations.

Notice that in CORBA, as in most distributed systems, remote bindings
are established implicitly without application intervention. This
provides the illusion that all objects are local, a property known as
`location transparency'. CORBA does not specify when such bindings
should be established or how they should be multiplexed over the
underlying network connections. Instead, ORBs are free to implement
implicit binding by a variety of means.

The rest of this chapter describes how omniORB manages network
connections and the programming interface to fine tune the management
policy.

\section{The Model}

omniORB is designed from the ground up to be fully multi-threaded. The
objective is to maximise the degree of concurrency and at the same
time eliminate any unnecessary thread overhead. Another objective is
to minimise the interference by the activities of other threads on the
progress of a remote invocation. In other words, thread `cross-talk'
should be minimised within the ORB. To achieve these objectives, the
degree of multiplexing at every level is kept to a minimum.

On the client side of a connection, the thread that invokes on a proxy
object drives the IIOP protocol directly and blocks on the connection
to receive the reply. On the server side, a dedicated thread blocks on
the connection. When it receives a request, it performs the up-call to
the object and sends the reply when the up-call returns. There is no
thread switching along the call chain.

With this design, there is at most one call in-flight at any time on a
connection. If there is only one connection, concurrent invocations to
the same remote address space would have to be serialised. To
eliminate this limitation, omniORB implements a dynamic
policy---multiple connections to the same remote address space are
created on demand and cached when there are concurrent invocations in
progress.

To be more precise, a network connection to another address space is
only established when a remote invocation is about to be made.
Therefore, there may be one or more object references in one address
space that refer to objects in a different address space but unless
the application invokes on these objects, no network connection is
made.

It is wasteful to leave a connection open when it has been left unused
for a considerable time. Too many idle connections could block out new
connections to a server when it runs out of spare communication
channels. For example, most Unix platforms have a limit on the number
of file handles a process can open. 64 is the usual default limit. The
value can be increased to a maximum of a thousand or more by changing
the `ulimit' in the shell.

\section{Idle Connection Shutdown and Remote Call Timeout}
\label{sec:shut}

Inside the ORB, a thread is dedicated to scan for idle connections.
The thread looks after both the outgoing connections and the incoming
connections.

When a connection is idle for a period of time, the connection is
shutdown. Similarly, if a remote call has not completed within a
defined period of time, the connection is shutdown and the ORB will
return \code{COMM\_FAILURE} to the client.

How often the internal thread scans the connections is determined by
the value of the \emph{scan granularity}. This value is defaulted to 5
seconds and can be changed using the command-line option
\texttt{-ORBscanGranularity}. Notice that this value determines the
precision the ORB is able to keep to the value of the idle connection
or remote call timeout.

How long the ORB will wait before it shuts down an idle connection is
determined by the idleConnectionPeriods. There are separate values for
incoming and outgoing connections. The default values are 180 and 120
seconds for incoming and outgoing connections respectively. These
values can be changed using the command-line options
\texttt{-ORBinConScanPeriod} and \texttt{-ORBoutConScanPeriod}.

Similarly, how long the ORB will wait for a remote call to complete is
determined by the parameter clientCallTimeOutPeriod for the client
side and the serverCallTimeOutPeroid for the server side. By default
calls will not timeout on either the client or server side.

The timeout can be changed with the
\texttt{-ORBclientCallTimeOutPeriod} and
\texttt{-ORBserverCallTimeOutPeriod} options. The scan can be disabled
completely by setting the scan granularity to 0.

\section{Interoperability Considerations}

The IIOP specification allows both the client and the server to
shutdown a connection unilaterally. When one end is about to shutdown
a connection, it should send a closeConnection message to the other
end. It should also make sure that the message will reach the other
end before it proceeds to shutdown the connection.

The client should distinguish between an orderly and an abnormal
connection shutdown. When a client receives a closeConnection message
before the connection is closed, the condition is an orderly shutdown.
If the message is not received, the condition is an abnormal shutdown.
In an abnormal shutdown, the ORB should raise a \code{COMM\_FAILURE}
exception whereas in an orderly shutdown, the ORB should \emph{not}
raise an exception and should try to re-establish a new connection
transparently.

omniORB implements these semantics completely. However, it is known
that some ORBs are not (yet) able to distinguish between an orderly
and an abnormal shutdown. Usually this is manifested as the client in
these ORBs seeing a \code{COMM\_FAILURE} occasionally when connected
to an omniORB server. The work-around is either to catch the exception
in the application code and retry, or to turn off the idle connection
shutdown inside the omniORB server.


\section{Connection Acceptance}
\label{sec:accept}

omniORB provides the hook to implement a connection acceptance policy.
Inside the ORB runtime, a thread is dedicated to receive new
connections. When the thread is given the handle of a new connection
by the operating system, it calls the policy module to decide if the
connection can be accepted. If the answer is yes, the ORB will start
serving requests coming in from that connection. Otherwise, the
connection is shutdown immediately.

There can be a number of policy module implementations. The basic one
is a dummy module which just accepts every connection.

In addition, a host-based access control module is available on Unix
platforms. The module uses the IP address of the client to decide if
the connection can be accepted. The module is implemented using
\emph{tcp\_wrappers 7.6}. The access control policy can be defined as
rules in two access control files: \file{hosts.allow} and
\file{hosts.deny}. The syntax of the rules is described in the manual
page \file{hosts_access(5)} which can be found in
appendix~\ref{apx:hostsaccess}. The syntax defines a simple access
control language that is based on client (host name/address, user
name), and server (process name, host name/address) patterns. When
searching for a match on the server process name, the ORB uses the
value of \code{omniORB::serverName}. \op{ORB\_init} uses the argument
\code{argv[0]} to set the default value of this variable. This can be
overridden by the application with the \texttt{-ORBserverName
}<\textit{string}> command line argument

The default location of the access control files is \file{/etc}. This
can be overridden by the extra options in \file{omniORB.cfg}. For
instance:

{\small
\begin{verbatim}
# omniORB configuration file - extra options

GATEKEEPER_ALLOWFILE   /project/omni/var/hosts.allow

GATEKEEPER_DENYFILE    /project/omni/var/hosts.deny

\end{verbatim}
}

As each policy module is implemented as a separate library, the choice
of policy module is determined at program linkage time. For instance,
if the host-based access control module is in use:

{\small
\begin{verbatim}
% eg1 -ORBtraceLevel 2
omniORB gateKeeper is tcpwrapGK 1.0 - based on tcp_wrappers_7.6 
I said,"Hello!". The Object said,"Hello!"
\end{verbatim}
}

\noindent Whereas if the dummy module is in use:

{\small
\begin{verbatim}
% eg1 -ORBtraceLevel 2
omniORB gateKeeper is not installed. All incoming are accepted.
I said,"Hello!". The Object said,"Hello!"
\end{verbatim}
}



\appendix
%%%%%%%%%%%%%%%%%%%%%%%%%%%%%%%%%%%%%%%%%%%%%%%%%%%%%%%%%%%%%%%%%%%%%%
\chapter{hosts\_access(5)}
%%%%%%%%%%%%%%%%%%%%%%%%%%%%%%%%%%%%%%%%%%%%%%%%%%%%%%%%%%%%%%%%%%%%%%
\label{apx:hostsaccess}

\subsection*{DESCRIPTION}

This manual page describes a simple access control language that is
based on client (host name/address, user name), and server (process
name, host name/address) patterns.  Examples are given at the end. The
impatient reader is encouraged to skip to the EXAMPLES section for a
quick introduction.

An extended version of the access control language is described in the
hosts\_\dsc{}options(5) document. The extensions are turned on at
program build time by building with -DPROCESS\_OPTIONS.

In the following text, \term{daemon} is the process name of a network
daemon process, and \term{client} is the name and/or address of a host
requesting service. Network daemon process names are specified in the
inetd configuration file.

\subsection*{ACCESS CONTROL FILES}

The access control software consults two files. The search stops
at the first match:

\begin{itemize}
\item Access will be granted when a (daemon,client) pair matches an
entry in the \file{/etc/hosts.allow} file.

\item Otherwise, access will be denied when a (daemon,client) pair
matches an entry in the \file{/etc/hosts.deny} file.

\item Otherwise, access will be granted.

\end{itemize}

A non-existing access control file is treated as if it were an empty
file. Thus, access control can be turned off by providing no access
control files.

\subsection*{ACCESS CONTROL RULES}

Each access control file consists of zero or more lines of text.
These lines are processed in order of appearance. The search
terminates when a match is found.

\begin{itemize}

\item A newline character is ignored when it is preceded by a
backslash character. This permits you to break up long lines so that
they are easier to edit.

\item Blank lines or lines that begin with a \texttt{\#} character are
ignored.  This permits you to insert comments and whitespace so that
the tables are easier to read.

\item All other lines should satisfy the following format, things
between [] being optional: \texttt{daemon\_list : client\_list [ :
shell\_command ] }

\end{itemize}

\texttt{daemon\_list} is a list of one or more daemon process names
(argv[0] values) or wildcards (see below).  

\texttt{client\_list} is a list of one or more host names, host
addresses, patterns or wildcards (see below) that will be matched
against the client host name or address.

The more complex forms \texttt{daemon@host} and \texttt{user@host} are
explained in the sections on server endpoint patterns and on client
username lookups, respectively.

List elements should be separated by blanks and/or commas.  

With the exception of NIS (YP) netgroup lookups, all access control
checks are case insensitive.

\subsection*{PATTERNS}

The access control language implements the following patterns:

\begin{itemize}

\item A string that begins with a \texttt{.} character. A host name is
matched if the last components of its name match the specified
pattern.  For example, the pattern \texttt{.tue.nl} matches the host
name \texttt{wzv.win.tue.nl}.

\item A string that ends with a \texttt{.} character. A host address
is matched if its first numeric fields match the given string.  For
example, the pattern \texttt{131.155.} matches the address of (almost)
every host on the Eindhoven University network (\texttt{131.155.x.x}).

\item A string that begins with an \texttt{\@} character is treated as
an NIS (formerly YP) netgroup name. A host name is matched if it is a
host member of the specified netgroup. Netgroup matches are not
supported for daemon process names or for client user names.

\item An expression of the form \texttt{n.n.n.n/m.m.m.m} is
interpreted as a `net/mask' pair. A host address is matched if `net'
is equal to the bitwise AND of the address and the `mask'. For
example, the net/mask pattern
\texttt{131.155.72.0/\dsc{}255.255.254.0} matches every address in the
range \texttt{131.155.72.0} to \texttt{131.155.73.255}.

\end{itemize}

\subsection*{WILDCARDS}

The access control language supports explicit wildcards:

\begin{description}
\item[\tt ALL]\mbox{}\\
The universal wildcard, always matches.

\item[\tt LOCAL]\mbox{}\\
Matches any host whose name does not contain a dot
character.

\item[\tt UNKNOWN]\mbox{}\\
Matches any user whose name is unknown, and matches
any host whose name or address are unknown.  This pattern should be
used with care: host names may be unavailable due to temporary name
server problems. A network address will be unavailable when the
software cannot figure out what type of network it is talking to.

\item[\tt KNOWN]\mbox{}\\
Matches any user whose name is known, and matches any
host whose name and address are known. This pattern should be used
with care: host names may be unavailable due to temporary name server
problems.  A network address will be unavailable when the software
cannot figure out what type of network it is talking to.

\item[\tt PARANOID]\mbox{}\\
Matches any host whose name does not match its
address.  When tcpd is built with -DPARANOID (default mode), it drops
requests from such clients even before looking at the access control
tables.  Build without -DPARANOID when you want more control over such
requests.

\end{description}

\subsection*{OPERATORS}

\begin{description}

\item[\tt EXCEPT]\mbox{}\\
Intended use is of the form: \texttt{list\_1} \texttt{EXCEPT}
\texttt{list\_2}; this construct matches anything that matches
\texttt{list\_1} unless it matches \texttt{list\_2}.  The
\texttt{EXCEPT} operator can be used in \texttt{daemon\_lists} and in
\texttt{client\_lists}. The \texttt{EXCEPT} operator can be nested: if
the control language would permit the use of parentheses, \texttt{a
EXCEPT b EXCEPT c} would parse as \texttt{(a EXCEPT (b EXCEPT c))}.

\end{description}

\subsection*{SHELL COMMANDS}

If the first-matched access control rule contains a shell command,
that command is subjected to \texttt{\%<letter>} substitutions (see
next section).  The result is executed by a /bin/sh child process with
standard input, output and error connected to /dev/null.  Specify an
\texttt{\&} at the end of the command if you do not want to wait until
it has completed.

Shell commands should not rely on the PATH setting of the inetd.
Instead, they should use absolute path names, or they should begin
with an explicit \texttt{PATH=\dsc{}whatever} statement.

The hosts\_options(5) document describes an alternative language that
uses the shell command field in a different and incompatible way.

\subsection*{\% EXPANSIONS}

The following expansions are available within shell commands:

\begin{itemize}

\item[\tt \%a (\%A)] The client (server) host address.
\item[\tt \%c] Client information: user@host, user@address, a host
name, or just an address, depending on how much information is
available.
\item[\tt \%d] The daemon process name (argv[0] value).
\item[\tt \%h (\%H)]
The client (server) host name or address, if the host name is
unavailable.
\item[\tt \%n (\%N)] The client (server) host name (or "unknown" or
"paranoid").
\item[\tt \%p] The daemon process id.
\item[\tt \%s] Server information: daemon@host, daemon@address, or
just a daemon name, depending on how much information is available.
\item [\tt \%u] The client user name (or "unknown").
\item [\tt \%\%] Expands to a single \texttt{\%} character.

\end{itemize}

Characters in \% expansions that may confuse the shell are replaced by
underscores.

\subsection*{SERVER ENDPOINT PATTERNS}

In order to distinguish clients by the network address that they
connect to, use patterns of the form:

\texttt{process\_name@host\_pattern : client\_list ... }

Patterns like these can be used when the machine has different
internet addresses with different internet hostnames.  Service
providers can use this facility to offer FTP, GOPHER or WWW archives
with internet names that may even belong to different
organisations. See also the `twist' option in the hosts\_options(5)
document. Some systems (Solaris, FreeBSD) can have more than one
internet address on one physical interface; with other systems you may
have to resort to SLIP or PPP pseudo interfaces that live in a
dedicated network address space. The \texttt{host\_pattern} obeys
the same syntax rules as host names and addresses in
\texttt{client\_list} context. Usually, server endpoint information is
available only with connection-oriented services.


\subsection*{CLIENT USERNAME LOOKUP}

When the client host supports the RFC 931 protocol or one of its
descendants (TAP, IDENT, RFC 1413) the wrapper programs can retrieve
additional information about the owner of a connection. Client
username information, when available, is logged together with the
client host name, and can be used to match patterns like:

\texttt{daemon\_list : ... user\_pattern@host\_pattern ...}

The daemon wrappers can be configured at compile time to perform
rule-driven username lookups (default) or to always interrogate the
client host.  In the case of rule-driven username lookups, the above
rule would cause username lookup only when both the
\texttt{daemon\_list} and the \texttt{host\_pattern} match.

A user pattern has the same syntax as a daemon process pattern, so the
same wildcards apply (netgroup membership is not supported).  One
should not get carried away with username lookups, though.

\begin{itemize}

\item The client username information cannot be trusted when it is
needed most, i.e. when the client system has been compromised.  In
general, ALL and (UN)KNOWN are the only user name patterns that make
sense.

\item Username lookups are possible only with TCP-based services, and
only when the client host runs a suitable daemon; in all other cases
the result is `unknown'.

\item A well-known UNIX kernel bug may cause loss of service when
username lookups are blocked by a firewall. The wrapper README
document describes a procedure to find out if your kernel has this
bug.

\item Username lookups may cause noticeable delays for non-UNIX users.
The default timeout for username lookups is 10 seconds: too short to
cope with slow networks, but long enough to irritate PC users.

\end{itemize}

Selective username lookups can alleviate the last problem. For
example, a rule like:

\texttt{daemon\_list : @pcnetgroup ALL@ALL }

would match members of the pc netgroup without doing username lookups,
but would perform username lookups with all other systems.

\subsection*{DETECTING ADDRESS SPOOFING ATTACKS}

A flaw in the sequence number generator of many TCP/IP implementations
allows intruders to easily impersonate trusted hosts and to break in
via, for example, the remote shell service.  The IDENT (RFC931 etc.)
service can be used to detect such and other host address spoofing
attacks.

Before accepting a client request, the wrappers can use the IDENT
service to find out that the client did not send the request at all.
When the client host provides IDENT service, a negative IDENT lookup
result (the client matches \texttt{UNKNOWN@host}) is strong evidence
of a host spoofing attack.

A positive IDENT lookup result (the client matches
\texttt{KNOWN@host}) is less trustworthy. It is possible for an
intruder to spoof both the client connection and the IDENT lookup,
although doing so is much harder than spoofing just a client
connection. It may also be that the client's IDENT server is lying.

Note: IDENT lookups don't work with UDP services. 

\subsection*{EXAMPLES}

The language is flexible enough that different types of access control
policy can be expressed with a minimum of fuss. Although the language
uses two access control tables, the most common policies can be
implemented with one of the tables being trivial or even empty.

When reading the examples below it is important to realise that the
allow table is scanned before the deny table, that the search
terminates when a match is found, and that access is granted when no
match is found at all.

The examples use host and domain names. They can be improved by
including address and/or network/netmask information, to reduce the
impact of temporary name server lookup failures.

\subsection*{MOSTLY CLOSED}

In this case, access is denied by default. Only explicitly authorised
hosts are permitted access.

The default policy (no access) is implemented with a trivial deny
file:

{\small
\begin{verbatim}
/etc/hosts.deny:
    ALL: ALL
\end{verbatim}
}

This denies all service to all hosts, unless they are permitted access
by entries in the allow file.

The explicitly authorised hosts are listed in the allow file.  For
example:

{\small
\begin{verbatim}
/etc/hosts.allow:
   ALL: LOCAL @some_netgroup
   ALL: .foobar.edu EXCEPT terminalserver.foobar.edu
\end{verbatim}
}

The first rule permits access from hosts in the local domain (no .  in
the host name) and from members of the \texttt{some\_netgroup}
netgroup.  The second rule permits access from all hosts in the
\texttt{foobar.edu} domain (notice the leading dot), with the
exception of \texttt{terminalserver.foobar.edu}.

\subsection*{MOSTLY OPEN}

Here, access is granted by default; only explicitly specified hosts
are refused service.

The default policy (access granted) makes the allow file redundant so
that it can be omitted.  The explicitly non-authorised hosts are
listed in the deny file. For example:

{\small
\begin{verbatim}
/etc/hosts.deny:
   ALL: some.host.name, .some.domain
   ALL EXCEPT in.fingerd: other.host.name, .other.domain
\end{verbatim}
}

The first rule denies some hosts and domains all services; the second
rule still permits finger requests from other hosts and domains.


\subsection*{BOOBY TRAPS}

The next example permits tftp requests from hosts in the local domain
(notice the leading dot).  Requests from any other hosts are denied.
Instead of the requested file, a finger probe is sent to the offending
host. The result is mailed to the superuser.

{\small
\begin{verbatim}
/etc/hosts.allow:
   in.tftpd: LOCAL, .my.domain

/etc/hosts.deny:
   in.tftpd: ALL: (/some/where/safe\_finger -l @%h | \
       /usr/ucb/mail -s %d-%h root) &
\end{verbatim}
}

The \texttt{safe\_finger} command comes with the tcpd wrapper and
should be installed in a suitable place. It limits possible damage
from data sent by the remote finger server.  It gives better
protection than the standard finger command.

The expansion of the \%h (client host) and \%d (service name)
sequences is described in the section on shell commands.

Warning: do not booby-trap your finger daemon, unless you are prepared
for infinite finger loops.

On network firewall systems this trick can be carried even further.
The typical network firewall only provides a limited set of services
to the outer world. All other services can be "bugged" just like the
above tftp example. The result is an excellent early-warning system.

\subsection*{DIAGNOSTICS}

An error is reported when a syntax error is found in a host access
control rule; when the length of an access control rule exceeds the
capacity of an internal buffer; when an access control rule is not
terminated by a newline character; when the result of %<letter>
expansion would overflow an internal buffer; when a system call fails
that shouldn\'t.  All problems are reported via the syslog daemon.


\subsection*{FILES}

\noindent \file{/etc/hosts.allow}, (daemon,client) pairs that are granted access.

\noindent \file{/etc/hosts.deny}, (daemon,client) pairs that are denied access.

\subsection*{SEE ALSO}

\noindent tcpd(8) tcp/ip daemon wrapper program.

\noindent tcpdchk(8), tcpdmatch(8), test programs.

\subsection*{BUGS}

If a name server lookup times out, the host name will not be available
to the access control software, even though the host is registered.

Domain name server lookups are case insensitive; NIS (formerly YP)
netgroup lookups are case sensitive.

\subsection*{AUTHOR}

Wietse Venema (wietse@wzv.win.tue.nl)\\
Department of Mathematics and Computing Science\\
Eindhoven University of Technology\\
Den Dolech 2, P.O. Box 513,\\
5600 MB Eindhoven, The Netherlands\\


\backmatter

\bibliography{omniORBpy}

\end{document}
