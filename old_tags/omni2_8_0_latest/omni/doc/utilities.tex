\documentclass[11pt,twoside,onecolumn]{article}
\usepackage[]{fontenc}
\usepackage{palatino}
\usepackage{a4}
\addtolength{\oddsidemargin}{-0.2in}
\addtolength{\evensidemargin}{-0.6in}
\addtolength{\textwidth}{0.5in}
\pagestyle{headings}

\title{The omniORB utilities}

\author{Eoin Carroll\\
Olivetti \& Oracle Research Laboratory\\
Cambridge}

\date{\today}

\begin{document}

\maketitle

\section{catior}
Usage:
 \begin{verbatim} catior [-x] <stringified IOR> \end{verbatim}

catior is a utility for viewing components of a stringified IOR.
It displays the components of the stringified object reference
supplied to it. 

The options are:


\begin{tabular}{ll}
\verb.-x. \\
  & Display the object key in hexadecimal.
\end{tabular}


\section{genior}
Usage: 
\begin{verbatim}genior [-x] <Type ID> <hostname> <port number> [object key] \end{verbatim}

genior generates a stringified object reference
from the arguments supplied to it.
If an object key argument isn't supplied, it will
use an object key generated by omniORB2.

The options are:


\begin{tabular}{ll}
\verb.-x. \\
 & Interpret the object key as a hexadecimal value. This 
value should begin with "0x"
\end{tabular}


\section{nameclt}
Usage: 
\begin{verbatim}nameclt [-ior <object-reference>] [-advanced] <operation> 
\end{verbatim}

The nameclt command invokes operations on the Naming Service.  

\subsection{Operations}

The allowed operations are:

\begin{description}
\item \verb.list  <context-name>.
  \subitem lists contexts and objects bound in the context with the specified name.
\item \verb.bind_new_context <context-name>.
  \subitem binds name to a new context, and returns the stringified context IOR.
\item \verb.remove_context <context-name>.
  \subitem unbinds and destroys the named context, as long as it is empty.
\item \verb.bind <object-name> <stringified-IOR>.
  \subitem binds name to object.
\item \verb.unbind <object-name>.
  \subitem unbinds name and object.
\item \verb.resolve  <object-name>.
  \subitem returns stringified IOR bound to specified name.

\end{description}


\subsection{Options}

The options are:

\begin{description}

\item \verb.-ior <NameService-object-reference>.
  \subitem  
Use the given stringified IOR as the "root" context of the naming service.  By
default, nameclt uses the object reference returned by calling: \\
\verb.CORBA::ORB::resolve_initial_references("NameService").

\item \verb.-advanced.
\subitem
Allow advanced operations.  These are operations which should not normally need
to be used.  They may however be useful for testing the naming service and also
for cleaning up in the event of a client messing up the namespace.  The
operations are:

\subitem \verb.bind_context <context-name> <stringified-IOR>.
\subsubitem binds name to context.
\subitem \verb.rebind <object-name> <stringified-IOR>.
\subsubitem binds name to object even if binding already exists.
\subitem \verb.rebind_context <context-name> <stringified-IOR>.
\subsubitem binds name to context even if binding already exists.
\subitem \verb.new_context.
\subsubitem returns stringified IOR for a new context.
\subitem \verb.destroy.
\subsubitem destroys the naming context given with \verb.-ior. flag.

\end{description}

\end{document}
